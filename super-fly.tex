% Maybe save the critiques for the Claudine chapter. Allude to it but save the meat of it for the next chapter


\section{Introduction}


The white, homogenous perspective of the United States, its history, and Turneresque identity that I have argued was inherent in the Western genre partly inspired a spate of action films that emerged in the early 1970s to challenge this narrative and ideological framing.
This genre was termed blaxploitation–a portmanteau of ``black" and ``exploitation"–and presented a vision of America that was ostensibly oppositional to the typical John Wayne film, though it maintained several similarities in theme, narrative, and ideology.
While Westerns were essentially period films set in rural locations, with overwhelmingly white casts, and featuring traditional, orchestral scores, blaxploitation films were typically set in contemporary inner-cities, starred black protagonists, and featured soundtracks of modern soul, funk, and R\&B.
I now turn my attention to one of the most celebrated of these films: \textcite{parks_jr_super_1972}.
\textcite{parks_jr_super_1972}'s success supposedly inspired the coining of term ``blaxploitation," and follows Youngblood Priest, a successful drug dealer in Harlem, as he attempts to make one last major deal before he is able to walk away from his life of crime and begin anew.




\section{Super Fly's Score}

\begin{table}
    \centering
    \begin{tabular}{>{\centering\arraybackslash}p{2.75cm}>{\centering\arraybackslash}p{2.25cm}>{\centering\arraybackslash}p{2cm}>{\centering\arraybackslash}p{1.5cm}>{\centering\arraybackslash}p{2cm}>{\centering\arraybackslash}p{2.25cm}}\toprule
         Release&   Chart&Debut Date&  Chart Peak&  Peak Date&  RIAA Certification (date)\\\midrule
         Super Fly (album)&    US Pop Albums&26 August 1972&  1&  21 October 1972&  Gold (7 September 1972)\\
         Freddie's Dead (Theme from Super Fly) (single)&   US Pop Singles&19 August 1972&  4&  4 November 1972&  Gold (31 October 1972)\\
 & US Black Singles& 12 August& 2& 7 October 1972& \\
         Superfly (single)&   US Pop Singles&18 November 1972&  8&  13 January 1973&  Gold (18 January 1973)\\
 & US Black Singles& 25 November 1972& 5& 13 January 1973& \\ \bottomrule 
    \end{tabular}
    \caption{Overview of the notable chart success of Curis Mayfield's soundtrack for \textit{Super Fly}.}
    \label{tab:super-charts}
\end{table}
\autocites[][]{noauthor_gold_nodate}


I will discuss \textit{Super Fly}'s themes in depth below, but before I do so it is importance to note the significance of the score to the film's legacy and its contemporary commercial success.
Composed, performed, and produced by Curtis Mayfield, \textit{Super Fly}'s soundtrack album featured nine original songs, three of which were instrumentals.
Scholar Mark Anthony Neal writes that \textit{Super Fly}, along with another key blaxploitation text, \textcite{parks_shaft_1971}, ``offered genre-bending fusions of soul, funk, and contemporary jazz, with commercial appeal."
Efforts at this commercial appeal is evidenced in the multiple formats the soundtrack was made available in–12" LP, 8-track tape, cassette tape, and 7" singles–while Mayfield recorded spoken advertisements for radio commercials.
The soundtrack subsequently achieved great commercial and chart success, detailed in Table \ref{tab:super-charts}.

The film's promotional material makes obvious how much emphasis was placed on the soundtrack in order to draw audiences to see the film in theatres.
The film's pressbook, for example, features multiple promotional posters, almost all of which entice audiences to ``See and hear CURTIS MAYFIELD play his Super Fly score," referring to one scene in which Mayfield and his band are prominently featured performing in a bar frequented by the film's protagonists (Figure \ref{fig:super-lobby-card}).\autocite[][]{noauthor_press_1972}
The pressbook features multiple written articles, intended to be placed in local newspapers, with one article specifically focused Mayfield's score.
This article, titled ``The Mayfield Experience Reaches the Screen," notes that \textit{Super Fly} is ``marked by a whole gamut of firsts," and argues that the most important of these is Mayfield's first time on screen.\autocite[][2]{noauthor_press_1972}
This same article goes on to describe the score as a ``counter balance" and ``critical commentary" of the film's narrative, providing a key thematic dissonance which I will discuss below.
This pressbook thus encapsulates the conscious effort by the filmmakers and the film's distribution company, Warner Brothers, to use the soundtrack as a key aspect of their promotional campaign.

\begin{figure}
    \centering
    \includegraphics[width=1\linewidth]{img/super-press-book.pdf}
    \caption{The lobby card promoting \textit{Super Fly} and, notably, Curtis Mayfield's performance in the film.}
    \label{fig:super-lobby-card}
\end{figure}

The soundtrack album was released almost a month before the film, on 11 July 1972.
As Todd Mayfield wrote in his biography of his father, this was an ``odd way to orchestrate a release," but was essential in promoting such a low budget film.\autocite[][219]{mayfield_traveling_2017}
He adds that this was one of the deciding factors for writer Philip Fenty and producer Sig Shore when deciding to approach Mayfield to score the film.
Having such an establishing musician lent a legitimacy to the project that it might not have otherwise been afforded, and this in turn helped Fenty and Shore source funding from multiple private individuals.

The importance of the score to the film's general aesthetic, meanwhile, is evidenced in its featuring in 48 minutes of the film's 1 hour and 31 minute runtime (slightly over 52\% of the film's duration).
This included ten original songs, three of which were instrumentals, each played multiple times and sometimes in their entirety.\footnote{One of these pieces, the instrumental ``Militant March," did not appear on the original soundtrack release. It was not available on record until the 25th anniversary deluxe edition was released in 1997, despite appearing twice in the film.}
% Perhaps the most discussed instance of Mayfield's soundtrack is during a montage sequence which shows the extent of Priest's drug dealing empire; 
These songs each follow generic conventions of soul, funk and R\&B, with heavy percussion, rhythmic guitars played through wah-pedals, and string and horn arrangements.
As Mervyn Cooke notes in his \textit{A History of Film Music}, popular music has been used as an ``ethnic marker" in both music and film industries, and the increase in ``black-oriented" films meant an increase in soundtracks written and performed by black artists.\autocite[Cooke writes that, following the threat of legal action by the National Association for the Advancement of Colored People unless more people of colour were hired in the film industry, by 1972 roughly 25\% of US-produced films were ``black-oriented."][401]{cooke_history_2008}
Following Cooke's assertion, the rise of blaxploitation somewhat inevitably led to the rise of soundtracks employing black popular music traditions.

% Such instrumentation was typical for the blaxploitation genre.

For many, Isaac Hayes' score for \textcite{parks_shaft_1971} was key in establishing this template in blaxploitation.
The critical zenith of this scoring convention came in 1972 when Hayes won the Academy Award for best original song for his song ``Theme From Shaft."
However, he states that despite this success and influence, he was ``knocked by some critics at that time saying, `Who does he think he is, taking horns and strings and putting them on top of funky rhythm tracks?’"\autocite[][]{noauthor_isaac_nodate}

Of most significance is the lyrical content of each of the songs, most of which comment directly on the film's narrative.
Songs such as ``Freddie's Dead" and ``Eddie You Should Know Better" both address key characters, while ``Little Child Runnin' Wild," ``Pusherman," and ``Superfly" each touch upon the societal impact of the protagonists' cocaine enterprise.\footnote{Although the film and soundtrack album is written ``\textit{Super Fly,"} the single is stylised as ``Superfly."}
In doing so, these songs detail the detrimental effects that Priest's drug dealing has on his community, while also looking at the socio-economic situations that lead individuals into dealing and taking drugs.
Mayfield himself noted that this was a conscious effort on his part to provide a counterpart to the film's apparent glorification of criminal activity, which he worried was essentially a ``cocaine infomercial."\autocite[Curtis Mayfield, quoted in][212]{mayfield_traveling_2017}

The soundtrack's critical commentary on the unfolding narrative led Travis Atria, who co-wrote Mayfield's biography along with Mayfield's son Todd, to describe it as ``a personal manifesto, a passionate cry, and an incisive sociological statement on race in America."\autocite[][]{atria_curtis_2022}
While this was not Mayfield's first soundtrack, it is his most celebrated, and the most critically and commercially successful album he released as both a solo artist and member of his previous band, the Impressions.




% Album was released before film so songs may have been familiar (album: 11 July; film 4 August)






\subsection{Curtis Mayfield}

% What KINDS of films with he do
% What does he mean in the soundtrack/film world
% He knew importance of capitalism “not only [for] … financial stability. It also meant that others would not have the power to dilute his artistic message” - Schreiber



Curtis Mayfield was offered the opportunity to score \textit{Super Fly} by Philip Fenty and Sig Shore.
He saw this as an opportunity to build upon the success he had achieved thus far in his career, while he also recognised in the script many of the figures he grew up around during his childhood in Chicago, and saw the importance of depicting this environment in film.

Prior to scoring \textit{Super Fly}, Mayfield found success as the principal songwriter of the soul group the Impressions.
The group formed in 1958 and, following multiple personnel changes, he became the group's lead vocalist.
Prior to Mayfield's departure in 1970, the group released six albums that reached the top five on the R\&B Billboard charts and five number one singles on the R\&B singles chart.\autocite[][]{noauthor_impressions_2013}
As the 1960s progressed and calls for racial equality grew louder, Mayfield's lyrical references to social issues became more prevalent and overt.
For musicologist Tammy L. Kernodle, the Impressions were a key group in soul music's evolution away from ``sacred-based songs to a more secular genre."\autocite[][308]{kernodle_i_2008}
Throughout his career in the Impressions, as a solo artist, and as a composer for film, Mayfield drew upon soul, gospel, R\&B, and funk musics, and, as a 1980 \textit{Ebony} magazine article notes, was ``as at home in front of a 72-piece film orchestra as he [was] with a six-piece jazz combo."\autocite[][138]{norment_unknown_1980}

After leaving the Impressions in 1970, Mayfield released two solo studio albums prior to scoring \textit{Super Fly}: \textit{Curtis} (1970) and \textit{Roots} (1971) which reached number one and number six on the R\&B album charts, respectively.
His solo music grew more ``hard-hitting" in its political commentary, as music journalist Richie Unterberger states.\autocite[Richie Unterberger, quoted in][104]{schreiber_music_2020}
In this, musician Boz Scaggs wrote in 2005, Mayfield became a ``voice of activism, calling out the diseases of urban America."
Mayfield's activism often tacitly preached the importance of collectivism to overcome societal oppression, evident in the Impressions' 1967 single ``We're a Winner."\autocite[This message clearly resonated with audiences, as the single topped Billboard's R\&B chart and reached number 14 on the pop chart. The phrase ``We're a Winner" later became the motto for Mayfield's Curtom record label.][]{noauthor_were_nodate}
The celebration of societal collectivism is a quintessential aspect of soul which, Nicholas Forster writes, ``emphasizes collective possibility through a singularly empowered voice that shapes a shared intimacy between many."\autocite[][69]{forster_dont_2015}
This ideology can be heard in \textit{Super Fly}, in contrast to the ideology of the protagonist Youngblood Priest, who remains solely focused on accruing his own independence and financial success.

In his own life, Mayfield similarly attempted to maintain his own creative and financial autonomy, noting on multiple occasions that ``the important thing is to own as much of yourself as you can ... when you're even with \textit{the man}, you can call your own shots."\autocite[][59]{berry_curtis_1973}
This mindset led to him co-founding Curtom Records with Chicago music promoter and manager Eddie Thomas. 
The Impressions moved to this new label, giving Mayfield full control over their songs, and it was through this label that he released \textit{Super Fly}.
Music journalist Kris Needs writes that, by structuring Curtom in a way that gave him full control over the music and business, ``the label was a Black Power move in itself."
Needs' claim echoes Mayfield's son Todd, who wrote in his father's biography that Mayfield had ``done what a black man in America wasn't supposed to do–snatched control from a system designed to subjugate him."\autocite[][141]{mayfield_traveling_2017}


Following the success of \textit{Super Fly}, Mayfield went on to score a number of other films and found himself in great demand.
He commented upon the number of his film scores during this period:
\begin{quote}
I was standing in Chicago right on State Street, the main street in the Chicago theater district, right there in the Loop. And I looked out, and right there I could see the marquee for three of my movies at the same time, Super Fly, Let’s Do It Again, and I think it was Claudine. Right there in my hometown.\autocite[][216]{werner_higher_2004}
\end{quote}
Curtom's financial success also led to Mayfield investing in the film \textcite{young_short_1977}, which he both acted in and scored.
However, the film performed poorly and, according to Michael A. Gonzales, ultimately ``bankrupted" Curtom which was closed in 1980.\autocite[][]{gonzales_gangster_2020}

Mayfield is a important figure when considering 1970s film scores, and his soundtrack for \textit{Super Fly} remains one of the decade's most celebrated.\autocites[In a 2019 ``Best Movie Soundtracks of All Time" list, \textit{Pitchfork} named \textit{Super Fly} at number one, while \textit{Rolling Stone} named it the seventy sixth best album of all time. Also in 2019, it was added to the US Library of Congress's National Recording Registry marking it as "culturally, historically, or aesthetically significant."][]{pitchfork_50_2019}[][]{rolling\_stone\_500\_2023}[][]{noauthor\_new\_2019}
His drive to maintain commercial and creative control over his music demonstrates an adherence to Turner's American identity.
However, his personal politics implicitly critiques many of the traits Turner celebrated.
This is perhaps most notable in his celebration of collectivism over individualism and his focus on racial equality, which Turner dismissed.
He also more subtly touched upon issues of gender inequality, for example in his song ``Miss Black America."
% \textcolor{red}{THIS CAME AT A TIME WHEN BLACK POWER MOVEMENT WAS OVERLY FOCUSED ON MASCULINITY, AND FEMINIST MOVEMENT WAS OVERLY FOCUSED ON WHITE EXPERIENCE.}
This song is significant as it highlights that \textit{Claudine} was not Mayfield's sole attempt at writing from and about a female perspective.
I will explore this more in my next chapter.

Mayfield's representation of a rebuke to a Turneresque identity remains even today, albeit implicitly.
For example, some critics have evoked Mayfield's personal politics and ideology when discussing the USA's current political climate.
Such articles cite the continued relevance of Mayfield's messages, for example Atria's retrospective review of \textit{Curtis} on the occasion of its fiftieth anniversary in which he writes that,
\begin{quote}
\textit{Curtis }is a master class on the effects of American racism and our refusal as a nation to come to terms with it. After half a century, it still speaks to the American experience with startling insight and damning truth. It is both a prediction of and an explanation of the police killings of George Floyd and Breonna Taylor (and too many others), the resulting protests and riots, and the rise of the Black Lives Matter movement.
    \end{quote}
Atria later cites Donald Trump as an analogue of Richard Nixon, whose presidency \textit{Curtis} was responding to, due both men's ``racist dog-whistles, [their] calls for law and order, [and] coddling of white supremacists."\autocite[][]{atria_curtis_2020}
The similarity between the two presidencies, and the despair that Trump elicits from many on the left, has inspired some to cite Mayfield's politics as an panacea to the Trump era.

Kris Needs, for example, opens his 2017 overview of Mayfield's career by stating that the political rise of Donald Trump can be seen as an attempt by the ``racist elements" of the United States to make ``up for lost time."\autocite[][]{needs_beautiful_2017}
He continues, noting that for many this moment in history ``reignit[ed] fears and memories of the marches, riots, killings and triumphs of the 1960s," before pivoting to discuss Mayfield as a key figure in protesting against the 1960s and 70s equivalent of such political situations.\autocite[][]{needs_beautiful_2017}
Vivien Goldman makes this connection more explicitly in a 2017 article published the day before Trump's first inauguration titled ``Why America Needs Curtis Mayfield Now More Than Ever."\autocite[][]{goldman_why_2017}

While these articles are simply hypothetical exercises in considering how Mayfield would response to today's political climate, they highlight Mayfield's reputation for politically active songwriting and ``poetic messages of hope and equality."\autocite[][]{needs_beautiful_2017}
As I will detail below, this is a trait that remained throughout his film soundtracks and makes him a key composer to examine when exploring representations of American identity.
As I detailed in my introduction, Donald Trump is in many ways a full embodiment of Turner's ideal American figure: a individualist entrepreneur who gives the impression of a strong, independent masculinity.
By so frequently being evoked as a contrast to Trump, Mayfield is positioned as the antithesis of the Turneresque figure.
His two scores that I explore are therefore important texts when examining representations and critiques of an inherently conservative American identity.
Of course, Mayfield simultaneously reflects Turner's advocacy for entrepreneurial independence, complicating the connection between the two ideologies, yet still he provides opportunity for exploring the overlap between an embrace and a rejection of Turner's thesis.






% And yet, as X notes, “Political commentary is not a distinguishing trait of Mayfield’s Hollywood films” 



% "Mayfield hit his creative peak in 1972 with Superfly, his soundtrack for the blaxploitation classic. It was inner-city life set to music." - \url{https://www.proquest.com/docview/252688349?pq-origsite=primo&accountid=10673&searchKeywords=gladys%20knight%20claudine&sourcetype=Newspapers}



\section{Blaxploitation Overview}



Prior to the emergence of blaxploitation as a significant and financially lucrative subgenre, the film industry relied heavily on racist archetypes in the depictions of black characters.
Film historian Donald Bogle discusses this in his book named for these racist tropes, \textit{Toms, Coons, Mulattoes, Mammies, \& Bucks}.
Bogle writes that black characters, when they were included in Hollywood films at all, were exploited for comedic effect by ``stressing Negro inferiority."\autocite[][4]{bogle_toms_2003}
There remained a parallel industry of black filmmakers although these ``race films" did not attract investment or interest from the mainstream Hollywood studios until the 1950s.
Significant progress with regards to representation seemed to be made when Sidney Poitier became the first black actor to become the year's top grossing actor with a series of three films released in 1967:
\textcite{clavell_sir_1967}, \textcite{jewison_heat_1967}, and \textcite{kramer_guess_1967}.

Still, many criticised Poitier for performing overly idealised and passive characters, what critic Clifford Mason describes as ``antiseptic, one-dimensional hero[es]."\autocite[][]{mason_why_1967}
Mason goes on to call for films that break from white-centric narratives and foreground black experiences:
\begin{quote}
Until the concern of movies is for the dignity, the manhood, the thinking of the Negro in his world, with it historical past, its turbulent present and its hopeful future, there can be no true portrait of the Negro and no true art. Whites may or may not play a part in this world; the crucial need however is for a break with the concept that the world is only white, and that the Negro exists only in the white man's view of him.\autocite[][]{mason_why_1967}
\end{quote}
This sentiment was echoed by Melvin Van Peebles, director of what is often cited as the first major blaxploitation work, \textcite{van_peebles_sweet_1971}, when asked which black-starring films he disliked: ``every damn last one of them."\autocite[Melvin Van Peebles, quoted in][13]{henderson_black_2024}

These attitudes, in essence, gave rise to the series of films that came to be known as blaxploitation.

% ``given a clean suit and a complete purity of motivation so that, like a mistreated puppy, he has all the sympathy on his side" (Mason)

% Despite these economics and the apparent exploitation of black audiences, 

Due to the initial lack of institutional support, many of the most celebrated blaxploitation films were conceived, created, and funded almost entirely by black artists and communities.
\textit{Super Fly}, for example, was funded in large part through private investments, while its lead actor Ron O'Neal discussed proactive attempts to combat the lack of representation in mainstream cinema and the exploitation of the labour of black actors: ``I think what you’re going to find happening with the black actor ... you’re going to find black actors forming production companies themselves to make black films independently of the Hollywood production system." (8:21) NOT SURE HOW TO CITE THIS.
To this point, the company Third World Cinema was established to provide opportunities to people of colour both in front and behind the camera.
I will discuss Third World Cinema in more detail in my later chapter on \textit{Claudine}, but it is important to add here that a lot of the crew who worked on \textit{Super Fly} came to the project through this company.

Blaxploitation narratives focused on quick-witted, charismatic, and highly-sexualised protagonists, who must overcome (typically white) representatives of illicit corporate and governmental power, often corrupt police and government officials or groups such as the Mafia.
The heroes outwit and outfight these figures by demonstrating their superior street smarts and capacity for violence thus overcoming the white supremacist structures that attempt to maintain power over their neighbourhoods.
Film critic Vincent Canby provides a useful summary of these narrative tropes in a 1976 \textit{New York Times} article, somewhat problematically titled ``Are Black Films Losing Their Blackness?"
Canby describes the genre as ``supercharged, bad‐talking, highly romanticized melodramas about Harlem superstuds, the pimps, the private eyes and the pushers who more or less singlehandedly make whitey's corrupt world safe for black pimping, black private‐eyeing and black pushing."\autocite[][]{canby_are_1976}

Canby's summary provides a useful context for Ed Guerrero's description of these films as ``narratives of Black empowerment."\autocite[][3]{guerrero_framing_1993}
These narratives of empowerment, musicologist Greil Marcus writes, was in part a response to a sentiment voiced by one character in \textcite{coppola_godfather_1972} when the heads of each of the Mafia families are discussing their heroin operation; they agree to distribute solely in Harlem because ``they're animals up there."\autocite[][90]{marcus_mystery_2000}
We can thus understand blaxploitation narratives as responses to films in which corrupt white power structures exert control and influence over communities such as Harlem.

With their assertive protagonists, willing and able to commit violence to ensure their freedom from societal and financial subjugation, these films reflected a philosophical shift away from the Civil Rights Movement's form of peaceful protest and toward a more aggressive demand for equality.
This sentiment was summarised by Student Nonviolent Coordinating Committee member Julius Lester in a 1966 paper:
\begin{quote}
Now it is over. America has had the chance to show that it really means ``that all men are endowed with certain inalienable rights." America has had previous chances in this decade to make it come true. Now it is over. The days of singing freedom songs and the days of combating billy clubs with Love ... For many black people, the time is imminent. For others it simply means the white man no longer exists. He is simply to be ignored, because the time has come for the black man to control the things which effect his life. Like the Irish control Boston, the black man will control Harlem.\autocites[][8-9]{lester_angry_1966}[this article was previously published in folk magazine \textit{Sing Out!} before being published by the Southern Student Organizing Committee][]{lester_angry_1966-1}
\end{quote}
Multiple blaxploitation films dramatise Lester's reference to controlling Harlem, often depicting individuals fighting for the neighbourhood in the face of oppressive structures, or otherwise focused on their personal status as they rise to powerful positions within criminal organisations.\footnote{In \textcite{davis_cotton_1970}, for example, two black detectives fight against a corrupt reverend exploiting their Harlem community.
Meanwhile, Harlem is the centre of a power struggle between the Italian Mafia and local black crime bosses in \textcite{shear_across_1972}. \textit{Super Fly}'s Youngblood Priest runs his Harlem drug empire and faces off against corrupt police officers who control the drug trade and therefore the neighbourhood.}

% Lester - page 3.
% How naive, how idealistic they were then. They had honestly believed that once white people knew what segregation did, it would be abolished. But why shouldn't they have believed it? They had been fed the American Dream, too. They believed in Coca Cola and the American Government. ``I dreamed I got my Freedom in a Maidenform bra." They were in the Pepsi Generation, believing that the F.B.I. was God's personal emissary to uphold good and punish evil.

Similar to how the Western came to define a certain type of masculinity, so did Blaxploitation narratives come to represent a specific image of African American male identity.
While there later were many female-led films, such as \textcite{hill_coffy_1973}, \textcite{hill_foxy_1974}, and \textcite{starrett_cleopatra_1973}, as Stephane Dunn notes, ``hypermasculine machismo [is] at the center of the genre” and ``patriarchal power and masculine bravado” was commonplace in the films' treatment of women.\autocite[][2]{dunn_baad_2008}
This hypermasculinity aligns the genre with the Western and its archetypal hero.
By extension, it reflect attributes detailed in Turner's thesis: blaxploitation heroes are individualist figures who reject attempts to curtail their autonomy; they demonstrate their masculine supremacy by dominating women; and their capitalist acumen allows them to gain power and profit.
The films' narratives also often follow similar plot points.
The titular hero of \textcite{parks_shaft_1971}, for example, is a tough and entrepreneurial private detective, hired to rescue the kidnapped daughter of a local mobster in a plot highly reminiscent of the classical Western \textcite{ford_searchers_1956}.
One sequence in \textit{Super Fly} sees Priest driving past the West End Theatre in Harlem which is advertising screenings of \textit{Big Jake} in further evidence of the ongoing popularity of Westerns.

In addition to these narrative similarities, a subgenre of blaxploitation Westerns also arose in this period.
Examples of this subgenre are \textcite{parks_jr_thomasine_1974}, \textcite{dawson_take_1975}, \textcite{spangler_joshua_1976}; however, these films did not achieve the commercial success that the contemporary, urban-set films did.
That so many of these films were released even after the Western had waned in popularity points to the enduring attraction of the Western hero, in addition to the longstanding legacy of Turner's vision of the American figure.


------------------------------------------------------------------

Despite their often meagre budgets, many of these films achieved tremendous box office success and have proven highly influential.
This influence can be seen in more mainstream contemporary films–perhaps most notably the James Bond film \textcite{hamilton_live_1973}–while the fashion, language, and music continues to inspire the aesthetic of many hip-hop artists.
For some, and perhaps more consequentially, the genre's legacy is that it helped to provide opportunities for people of colour in a film industry that notoriously did not often allow such diversity.
Pam Grier, one of the most celebrated female stars of the genre, notes this, explaining that ``there would not be African-American film-makers without Blaxploitation."\autocite[Pam Grier, quoted in][45]{noauthor_blaxploitation_1998}

And yet, there are many critics who cite the genre as exploitative of black artists and black audiences.
These criticisms stem from the involvement of major studios in the production and distribution of blaxploitation films after they had proven to be worthwhile financial investments.
% The genre has since come to be understood as an attempt by Hollywood production companies to take advantage of black audiences.
The ``exploitation" of blaxploitation therefore derives not only from the films' sensationalist excesses, but from corporations' exploitation of black audiences.
Melvin Van Peebles commented upon this in a 1998 genre retrospective in \textit{Uncut} magazine: ``what happened was \textit{Sweetback} made all this money and so the studios made the style into a caricature - they took out the militant part and that became Blaxploitation."\autocite[Melvin Van Peebles, quoted in][45]{noauthor_blaxploitation_1998}
This economic model was the target of one infamous critique published in the \textit{Journal of Black Studies and Research} in 1976 titled ``Black Films, White Profits."\autocite[][]{ward_black_1976}
In this paper, Reneé Ward explores findings from a recent study from \textit{The Chicago Reporter} on the socio-economic structures of the film industry, noting that between January 1973 and August 1974 in a central Chicago neighbourhood, ``black movies accounted for 41 per cent of the box office take."\autocite[][13]{ward_black_1976}
However, the majority of these profits ultimately went to the white owners of the theatres, production companies, and the films themselves.
Nevertheless, Craig McMullen, the guitarist who performed on the \textit{Super Fly} album, defends the genre for its economic benefits to black communities.
While he concedes that the genre did have exploitative traits, he adds that ``it seems like if you got that many people workin', I don't know who you exploiting. It seems like a black \textit{employment} situation to me."\autocite[Craig McMullen, quoted in][226]{mayfield_traveling_2017}

------------------------------------------------------------------



Perhaps the most important and oft-discussed aspects of blaxploitation films is their music.
These films typically featured soundtracks composed and performed by contemporary funk, soul, and R\&B artists, often boasting some of the biggest names in these genres, such as James Brown (\textcite{cohen_black_1973}), Marvin Gaye (\textcite{dixon_trouble_1972}), Earth, Wind \& Fire (\textcite{van_peebles_sweet_1971}), and Isaac Hayes (\textcite{parks_shaft_1971}).
Fred Williamson, who starred in several blaxploitation film, noted the significance of the films' soundtracks:
``the one thing about our movies was that they had dynamic music ... The music was as big as the movies.”\autocite[Fred Williamson, quoted in][]{gonzales_gangster_2020}
Of course, the incentive to use popular music artists came from the desire to increase a film's profitability, by using popular music to draw a wider audience while also profiting from the sale of singles and albums.
Composer Johnny Pate, who frequently collaborated with Curtis Mayfield and worked on the arrangements for \textit{Super Fly}, expressed his dissatisfaction with this overt commercialisation of film soundtracks, stating in a 1973 \textit{Billboard} interview that ``there are people making black films who are into a thing where they feel that if they get a song or a record or an artist involved, regardless of whether the movie is good or bad, they will have a hit.”\autocite[Johnny Pate, quoted in][]{robinson_soul_1973}
Matthew Tchepikova-Treon also discusses this trend wherein filmmakers and musicians would exploit the genre's formula to maximise profits, often at the expense of the film's socio-political significance: ``once music from a blaxploitation film was commercially objectified in the form of a soundtrack album it became merely symptomatic of late capitalism’s tyrannical hold over the totalizing effects of cultural production.”\autocite[][56]{tchepikova-treon_super_2020}

Further criticism of blaxploitation soundtracks focused on the 

Without their associated soundtracks, it is doubtful whether the blaxploitation genre would have achieved its commercial heights, regardless of such critiques.




Despite the genre's commercial peaks, the opportunities it provided those typically shut out of the film industry, and its influence on contemporary cinema and more recent music genres, for film historian Christopher Sieving ``few film critics or scholars have thought about the blaxploitation cinema in terms of an aesthetic legacy, with one major exception: Mayfield’s \textit{Super Fly} score."\autocite[][78]{sieving_super_2001}



% An interesting point about how Blaxploitation differed from contemporary paranoid, downer 70s films - they were a celebration!

\section{Super Fly}

\textit{Super Fly} was a foundational text in the burgeoning blaxploitation genre and became one of the genre's most profitable films, reportedly earning roughly \$18.6 million in 1972 from a budget of \$150,000.\autocite[The film's success notably led it to overtaking \textit{The Godfather}, the most profitable film in 1972, in the box office charts.][375-376]{barnett_super_2020}
The film was directed by Gordon Parks Jr., whose father had directed \textit{Shaft} the previous year.
Parks previously studied and worked as a photographer, a skillset he employed during one of the film's most famous sequences wherein a montage of still images shows Priest's employees selling to a range of customers.
While the film attracted criticism for supposedly glamourising criminal activity, Parks remained steadfast in his claim that the film was simply depicting the realities of urban life.
This defence seems to be backed up by the use of Mayfield's score in certain sequences, apparently condemning the onscreen actions it accompanies.




\subsection{Narrative Overview}

The film follows Youngblood Priest, a success Harlem cocaine dealer, and his partner Eddie.
Priest is portrayed from early in the film as ruthless and coldly focused on his money.
This is demonstrated in two early sequences, where he chases and savagely beats a drug user who attempted to rob in, and threatens one of his employees who fails to deliver his earnings by saying he will turn his wife into a prostitute.
He states his desire to make one final deal to earn him enough money to get out of the drug dealing business and to begin anew.
Eddie, however, has misgivings about the possibilities of escaping this life and repeatedly tries to convince Priest that their financial success makes up for the immoral means through which they achieved it.
Priest borrows money from his former mentor, Scatter, to purchase a large sum of cocaine and, in the montage of still images referenced above, his employees sell the drugs to earn Priest and Eddie \$1,000,000.

His success attracts the attention of corrupt police officers who insist Priest and Eddie continue dealing drugs on their behalf.
Priest realises that this makes it impossible for him to retire when Scatter is killed and he realises that the police responsible will kill him too when he is no longer useful to them.
He nevertheless plots his escape and is betrayed by Eddie.
He is apprehended by police officers and brought to Deputy Police Commissioner Reardon, who is revealed as the head of the drug operation.
Reardon again insists that Priest cannot retire and threatens to kill him; yet, Priest informs Reardon that he knew his identity all along and has put a contract on Reardon's and his family's life should anything happen to him.
Priest is then released by the police officers. 
The film ends with him driving away with his profits, having overcome ``the Man," as the camera tilts upward and zooms in on the top of the Empire State Building.




\subsection{Super Fly's Themes}

Much like many blaxploitation films, as I noted above, \textit{Super Fly} follows a male protagonist who in many ways fulfils the prototypical Turneresque figure most commonly associated with the Western genre. 
Priest is therefore similar in certain aspects to some John Wayne characters, for example Big Jake McCandles.
Both, for example, are quick to assert their dominance through violence, only respect other men when they have proven themselves suitably masculine, and express their disdain for what they view as effeminate traits in others.
The key difference, however, is in the moral righteousness of each character: Big Jake demonstrates a clear code of ethics, and is clearly portrayed as an admirable figure; Priest is an antihero, willing to cruelly punish his own friends and employees if they fail him.

This latter point is evidenced in an early sequence when Priest is informed that one of his employees, Freddie, has failed to deliver the money he owes.
Priest threatens Freddie by demanding the money or else he will force Freddie's wife into prostitution (00:12:39).
In this threat, and in the misogynistic language he employs, Priest demonstrates a clear sexism also inherent in Turner's thesis, in addition the blaxploitation genre more broadly.

% QUINN BOOK ON PIMP CULTURE, masculinity

% Capitalism and American Dream - Priest as entrepreneur, how capitalism is the only way to escape this life. Eddie cites the TV etc as a reason to stay.
% O'neal's comments on how no corporate overlords telling them what to do. So the film was only possible through a lack of capitalism.

% Individualism - Priest is an individ hero but O'Neal also notes how the film was possible by brothers and sisters all pitching it. Also through TWC which got jobs for POC, again pointing to the extratextual elements contrasting the film's themes.


\section{Soundtrack Analysis}

% Curtis's soft voice contrasts with gritty macho priest.

% FUNK AS BOTH A CELEBRATION OF IDENTITY/AMERICANISM - AND ALSO A CHALLENGE TO ONTOLOGICAL ESSENTIALISM, BY DRAWING ON AFRICAN ELEMENTS ETC, IT CHALLENGES THE NOTION OF A SINGLE NATIONAL IDENTITY THAT TURNER ARGUED FOR.

% neal, 67 - ``Superfly offered genre-bending fusions of soul, funk, and contemporary jazz, with substantial commercial value. Both recordings and films gave birth to a generation of soundtrack recordings that celebrate the unleashed black \textit{ubermensch} and his highly eroticized urban masculinity."
% idea of ubermensch - the composer with complete control, opposite of goldsmith and alien. Mayfield was the boss (?), not just a cog. And yet, it was still essentially ``a collection of singles". Contrast of capitalist structure and aims.



% "As an indigenous expression of the collective African American experience, [Soul music] served as a repository of racial consciousness [that transcended] the medium of entertainment [and] provided a ritual in song with which African Americans could identify and through which they could convey important in-group symbols." {Van Deburg)}
% Reggie Lavong: "like Blues, Soul music reflects, defines, and directs the strategies, expectations, and aspirations of black Americans."



% As musicologist Portia K. Maultsby demonstrates in her Timeline of African American Music created for Carnegie Hall.


\begin{table}
    \centering
    \begin{tabular}{>{\centering\arraybackslash}p{3.75cm}>{\centering\arraybackslash}p{3cm}>{\centering\arraybackslash}p{6cm}}\toprule
         Commentary Type&  Audience(s) & Key Characteristics\\\midrule
         Documentary&  Internal/External& Describes negative conditions designed to document the magnitude of problems and possible causes.\\
         Jeremiad&  External& Challenges outsiders to implement humanitarian beliefs and values.\\
         All God's Children&  External/Internal& Calls for equal treatment based on assertions of common interests and shared experiences.\\
         Defiant Challenge&  External& Demands that external forces cease and desist from exploitative behaviour.\\
         Awareness Raising Self-Criticism&  Internal& Describes negative conditions designed to broaden concern within the commuity.\\
         Collective Self-Help&  Internal/External& Calls for collective problem-solving efforts relying on local resources.\\
         Confrontational Declaration&  Internal/External& Advocates aggressive self-defense to confront direct and indirect manifestations of external control in black communities.\\
         Revolutionary Manifesto&  Internal/External& Calls for overturning existing political and economic institutions to advances liberation struggles.\\
         Spiritual Transcendence Exploration&  Internal/External& Advocates spiritual enlightenment to reduce social tension to heighten consciousness.\\ \bottomrule
    \end{tabular}
    \caption{James B. Stewart's ``Political Commentary Typologies in Black Popular Music." Reproduced from James B. Stewart, ``Message in the Music: Political Commentary in Black Popular Music from Rhythm and Blues to Early Hip Hop," \textit{The Journal of African American History} 90, no. 3 (Summer 2005): 204.}
    \label{tab:super-typologies-bpm}
\end{table}


Each of the tracks on \textit{Super Fly}'s soundtrack remain in the tradition of ``black popular musics."
The most prominent generic conventions used are soul and funk.
Both genres share a similar ancestry, drawing heavily on African-derived musical and cultural practices, and combining them with contemporary instrumentation and ideologies.
Prior to examining Mayfield's use of soul and funk, it is useful to provide a brief overview of these similar yet disparate genres.

As musicologist Portia K. Maultsby writes, soul music was a secular amalgam of gospel and rhythm \& blues, ``borrow[ing] the rhythms, musical structures, vocal and instrumental style, and emotional delivery of gospel music and transform[ing] rhythm and blues into soul."\autocite[][]{maultsby_timeline_nodate}
Soul music was inherently ingrained with socio-political significance, and its ``rise paralleled that of the Civil Rights and the Black Power movements, acting as the `soundtrack' for the ascension of Black pride during these years."\autocite[][]{maultsby_timeline_nodate}
Soul artists thus implicitly took the roles of ``politicians and spokesmen communicating the concept of `black pride.'"\autocite[][54]{maultsby_soul_1983}
This latter point is elucidated by Robert W. Stephens who writes in his historical overview of black popular music that ``in a broad sense, the philosophies and concepts of soul reflect the collective sensibilities of the black community."\autocite[][21]{stephens_soul_1984}
He returns to this summation in his concluding remarks: ``early soul reveals the themes of unity, ethnic consciousness, self-acceptance and awareness. From this philosophical model evolved a secular tradition rooted in past practices, but modified to fit current conditions."\autocite[][21]{stephens_soul_1984}

Funk took soul's political ideologies and incorporated more percussive and polyrhythmic elements.
Though Maultsby writes that funk was directly influenced by soul, in the introduction to his 2024 book \textit{The Funk Movement: Music, Culture, and Politics}, Reiland Rabaka posits that its origins date further back:
\begin{quote}
it could be argued that the origins of funk can be traced back to the Middle Passage, the horrific three-month-long voyages from Africa across the Atlantic Ocean that took place between the sixteenth and nineteenth centuries ... It was down in the odorous and ghastly depths of the ``slave ships,” surrounded by the stench of defecation, disease, and death, that funk – not merely funk music but funk culture, funk as a movement and form of rebellion against the established order – was born.\autocite[][7-8]{rabaka_funk_2024}
\end{quote}
During these voyages, enslaved Africans would sing songs from their respective cultures, using their chains to create (poly)rhythmic patterns.
``This was the first funk music," he continues, ``the melancholic music of the Middle Passage, and echoes of this music free-floated through 1960s and 1970s classic funk and its immediate aftermath in the rap music of the 1980s, 1990s, and beyond."\autocite[][8]{rabaka_funk_2024}

Both soul and funk therefore responded to their practitioners' contemporary environment and historic cultural heritage, while celebrating and encouraging collectivist ideologies and activism.
They also share significant socio-political meanings in their narratives and performances.
James B. Stewart provides a useful framework to understand these socio-political narratives in his ``political commentary typologies in black popular music" which can be seen in Table \ref{tab:super-typologies-bpm}.
I will refer to these typologies as I discuss certain songs from \textit{Super Fly} as it can help in understanding how Mayfield's songs pertain to ideas of a Turneresque nationalism.

The significance of the political ideology of black popular music is perhaps best understood with consideration of the Black Arts Movement, the ``cultural arts offshoot movement" of the Black Power Movement.\autocite[][19]{rabaka_funk_2024}
The political aims of the Black Arts Movement were defined by Larry Neal in his 1968 overview of the movement as ``radically opposed to any concept of the artist that alienates him from his community."\autocite[][29]{neal_black_1968}
He adds that the movement also strived for ``self-determination and nationhood."\autocite[][29]{neal_black_1968}

Self-determination is also a key facet of Turner's thesis, however his focus is primarily on how it pertains to the individual.
Neal, and the Black Arts Movement more broadly, instead address self-determination on larger, societal scale in response to generations of oppression.
Black popular music's approach to societal issues and sense of community therefore offers a contrasting perspective to the celebratory individualism discussed by Turner and heard in \textit{Big Jake} and similar Western scores.
The emphasis on collective autonomy and community-building inherent to the Black Arts Movement, and by extension soul and funk music, instils a collectivist ideology within Mayfield's soundtrack.
This may initially occur at a superficial, generic level.
Yet, by drawing upon a genre with strong ties to collective action, \textit{Super Fly} is tied to the notion of community over individuality and offers a tacit denouncement of Turner's thesis.
This is made explicit in many of the songs' lyrics wherein Mayfield laments characters' subjugation to the oppressive systems organised by the Man.

As I have alluded to above, black popular music isn inherently oppositional to the Turneresque American identity on an aesthetic and ideological level.
It is also ontologically oppositional, implicitly criticising Turner's notion of ontological essentialism.


\subsection{BPM's Ontological Roots/Routes}

Turner's suggestion that the American figure evolved from European settlers suggests that ``true" Americans are those with European ancestry, not those brought over from Africa or elsewhere.
His ideal American figure is thus implied as white and excludes people of colour.
Turner downplayed–or outright ignored–the significance of the presence of non-European identities by focusing solely on a rejection of \textit{European} cultural identity.

While a Turneresque approach to national identity means diluting diverse cultural and national backgrounds in search of a homogenous national identity, black popular music represents the diversity of the United States in a way that Western art music (WAM) does not.
This is evidenced in how black popular musics have been discussed as intrinsic to the American experience and imbued with similar ``American" signifiers as the music associated with the likes of Copland and Bernstein.
Jazz pianist Billy Taylor and jazz historian Grover Sales, for example, have both cited jazz as ``America's classical music," with Sales further describing it as a fusion of ``sound from every culture washed up on these shore."\autocites[][21]{taylor_jazz_1986}[][49]{sales_jazz_1992}

John H. McClendon III, a philosopher of African-American studies, discusses jazz as a uniquely ``American" practice in a paper that challenges the idea of a homogenous American identity:
``such judgments are predicated on a cultural philosophy that presumes the United States to be a singular national entity with a corresponding state apparatus, i.e., a nation-state."\autocite[][22]{mcclendon_iii_jazz_2004}
For McClendon, jazz–and by extension its various offshoots and affiliated subgenres–derived from the African American experience of slavery and inequality.
Rabaka likewise argues this in his aforementioned summation that funk originated during the Middle Passage.
These assertions challenge the very existence of a singular and definable nation identity.

Furthermore, these genres stand opposed to Turner's claim that ``American” should be defined in contrast to a white ``European” experience.
This is achieved through the fusing of modern instruments and ideologies with traditional, ancestral pan-African cultural influences and a conscious reckoning with the implications of an enforced diaspora.


% By drawing influence from non-European cultures, black popular music thus rejects Turner's notion that the American identity derives solely from a white, European experience.
% Rather than rejecting ancestral influence–in the way Turner suggests one should–these genres combine contemporary influences with ancestral experiences.
% These musics thus reject the very premise of a homogenous and easily definable national identity, and thereby stand in opposition to Turner's thesis.

Complicating notions of national identity in black popular music benefits from a brief discussion of Paul Gilroy's conception of the ``black Atlantic," the ``desire to transcend both the structures of the nation state and the constraints of ethnicity and national particularity.”\autocite[][19]{gilroy_black_1993}
Gilroy prioritises one's ``routes" over their ``roots," suggesting that black identity cannot be defined by its ontological ``root" but by understanding the ``traffic between African cultural forms and the political cultures of diaspora blacks over a long period."\autocite[][199]{gilroy_black_1993}
% Drawing upon Amiri Baraka–a key figure in the Black Arts Movement–Matthew P. Brown adds that ``African-American artists call on a tradition rooted in African religious worship; these practices are revised ... in the Christianized world of America."\autocite[][489]{brown_funk_1994}

With this understanding, both WAM and black popular music clearly reflect a uniquely American experience, yet told from contrasting perspectives.
As such, they represent the multiplicity of national identity in contrast to Turner's more simplistic, binary taxonomy.
The music heard in \textcite{sherman_big_1971}, for example, embraces and celebrates the American character codified by Turner.
Furthermore, in many Western films, the antagonists are essentially stand-ins for the threats facing ``civilised" frontier communities that challenge the white American's dominance.
By employing musical genres that defy simplistic national identifiers, and actively attempt to assert an alternative and more diverse national identity, \textit{Super Fly} presents an alternative conception of the American ideal.

However, this understanding perhaps unfairly imbues black popular music with a degree of passivity.
For Rabaka, these musics should be understood in more combative terms, as he writes that black popular music, specifically funk music, can be ``taken as a deconstruction of Western European and European American musical aesthetics."\autocite[][35]{rabaka_funk_2024}
He further describes the Black Arts Movement's ``conceptual core" as an ``incendiary effort to, literally, \textit{decolonise} every aspect of African American expressive culture."\autocite[][32, emphasis in original]{rabaka_funk_2024}
This perspective adds a degree of militancy to \textit{Super Fly}'s soundtrack, appropriately matching Priest's aggressive personality and attempts to assert himself as an individualist beholden to no one.
As I detailed above, it further mirrors the efforts of many filmmakers of colour to establish a parallel film industry independent of the predominately white, bourgeois Hollywood production system.



% Following Rabaka, black popular music can be understood as a challenge to Turner's formulation of American national identity despite similar efforts to shed European cultural identity.






% Yet, while the likes of Copland and Bernstein attempted to codify an American style of music, and their music became aligned with a film genre that often reflected Turner's ideology, black popular music was often (MOVED TO ABOVE)







% By drawing upon west African diasporic music traditions black popular music also rejects European cultures, but does so by challenging the notion that ``American” should be defined in contrast to ``European.”

% For Turner, the survival of the American frontiersmen required an evolution of their European ancestry.
% He thus acknowledged the influence of European cultures while insisting it was fundamentally ill-suited to the frontier environment.
% Quintessential ``American" composers, such as Aaron Copland, Roy Harris, and Elmer Bernstein, similarly used European cultures as a key influence in their codifying of an American sound.



% More importantly, it rejects the premise that ``American" should be understood as a binary identity and defined in opposition anything at all.





% In their acknowledgement of the influence of ancestral practices and their rejection of ontological essentialism, these writings denounce the very concept of Turner's thesis, that a codified ``American" identity can be found along the frontier.
% Black popular music further contradicts Turner's thesis and its potential to represent national identities by complicating one of its core concepts: the rejection of a European cultural identity.

% \subsection{Non-European}


% This contrasts with the music discussed in my previous study of \textcite{sherman_big_1971} and the Western genre, which drew upon European traditions and is often associated with a quintessential ``American" identity.
% Black popular music, on the other hand, can understood as a purposeful rejection of ``western"/European ideologies and practices, with Rabaka going so far as to suggest that ``funk music can be taken as a deconstruction of Western European and European American musical aesthetics."\autocite[][35]{rabaka_funk_2024}
% Rabaka has furthermore described the Black Arts Movement's ``conceptual core" as an ``incendiary effort to, literally, \textit{decolonise} every aspect of African American expressive culture."\autocite[][32, emphasis in original]{rabaka_funk_2024}




\subsection{Repetition}

As I outlined above, soul and funk share similar antecedents, socio-political frameworks, and many key musical aspects, such as instrumentation and structures.
Perhaps the most noticeable of musical aspects, and one that recurs throughout Mayfield's soundtrack, is the use of repeated motifs and chordal structures.
% For example, the first song heard in the film and one of the most frequently repeated songs, ``Little Child Runnin' Wild," repeats a simple two chord progression throughout.
% The song is in F\sharp minor and alternates between this tonic chord and its fourth, B minor, in a standard blues AABA structure.
% A bridge interrupts this pattern where the song remains solely on the F\sharp minor.
% The song features prominent rhythmic and percussive textures, and the arrangement is underpinned by a simple ascending bassline which repeats with very little variation throughout the song.
I will discuss the specific aspects of \textit{Super Fly}'s songs below, but here it is important to outline the ways that the concept of repetition has been understood in popular music studies and in black popular music.

The repetition of themes, motifs, and entire movements has of course long been a common aspect of Western Art Music, as well as other art forms, religious practices, and cultural traditions.
John McGrath discusses this in his study on repetition in the work of Samuel Beckett, noting that ``repetition is fundamental to music" and is ``the prime catalyst of the tonal Western art music tradition."\autocite[][37]{mcgrath_samuel_2018}
Richard Middleton, meanwhile, provides useful terminology to differentiate between different degrees of repetition in music.
Discursive repetition involves ``longer units" such as symphonic movements or full lyrical passages, and is supposedly more common in ``bourgeois" popular musics.\autocite[][269-270]{middleton_studying_1990}
On the other hand is musematic repetition which involves repeated musemes, better understood as ostinati or riffs.
Middleton claims that musematic repetition is typical of black popular music traditions and became more common in European and American popular musics only after the increased influence of African-derived cultural practices.\autocite[][275]{middleton_studying_1990}


% and while these genres do employ discursive repetition, these sections are built from smaller musemes.



Musematic and discursive repetition should not be seen as mutually exclusive, however, with black popular music often using discursive repetition, for example through repeated vocal melodies and the classic AABA blues structure.
It would be disingenuous to claim that this is due to the influence of WAM or ``bourgeois" traditions.
Any yet, even more disingenuous would be to cite repetition in black popular music as a shibboleth that inherently differentiates it from WAM, as some have attempted to do.
McGrath again discusses this, citing James A. Snead and Christopher Small, and notes that ``an essentialist dichotomy between linear Western and a `deliberately' circular African music is tenuous."\autocite[][36]{mcgrath_samuel_2018}

% SHOULD I INCLUDE A FIGURE OF LITTLE CHILD HERE TO HIGHLIGHT DIFFERENT TYPES OF REPETITION?

% Mayfield's \textit{Super Fly} musemes become obvious with close listening, and following Rose, it is easy to understand their repetition as representation not just of national identifiers but as critiques of political ideologies.

Beyond Middleton and the analysis of repetition from a musicological perspective, it has been a key point of philosophical consideration, with some questioning even the possibility of achieving pure repetition since with each repetition additional history, knowledge, and expectation is added.\autocite[John McGrath summarises such thoughts in his book \textit{Samuel Beckett, Repetition and Modern Music}: ``How can something be the same at a different point in time and context? While the echo of a motif might sound the same, its repositioning, or recontextualisation, nevertheless achieves difference."][33]{mcgrath_samuel_2018}
More pertinently for my purposes, cultural theorists such as Jacques Attali, Fredric Jameson, and Theodor Adorno have discussed repetition as a result of mass production within an increasingly industrialised culture wherein musical genres and styles have become standardised and follow established conventions.
Susan McClary summarises such critiques: art is a reflection of the artist's subjective self; structural and thematic progression signifies personal growth and development; by contrast, ``any reiteration registers as regression."\autocite[][457]{mcclary_rap_2017}
This perspective leads to the summation of repetition as inherently ``childish," which, in turn, extends to the deeply problematic understanding of African cultures as ``primitive" in binary opposition to the ``bourgeois" and ``sophisticated."\autocites[][268]{middleton_studying_1990}[Since repetition is often understood as predicated on the influence of pan-African musics, this of course leans into racist perceptions of African-derived cultures as unsophisticated, unintellectual, and inferior to European-derived bourgeois cultures. Simon Frith notes that the understanding of ``African music (and African-derived musics)" as more primitive and simplistic than European-derived music was not a reflection on the music necessarily, but rather as an extension of the racist notion that ```the African' is more primitive, more `natural' than the European." This led to the assumption that music associated with these cultures must also be simplistic.][127]{frith_performing_1996}


% and connotes a lack of subjectivity and individuality, each of which are supposedly inevitable results of mass industrialisation.\autocite[][268]{middleton_studying_1990}


% Over-reliance on repetition is thus caused by ``industrial standardization and represents a move toward a single totalitarian code."\autocite[][71]{rose_black_1994}


Tricia Rose, however, pushes back against these racist perspectives and the framing of repetition in black popular music within a capitalist lens.
For Rose, repetition in black popular music can instead be understood as a means to challenge ideological structures and mass commodification, albeit within, and using the tools of, mass culture.
In this context, mass commodification refers to the ``institutionalized ... expectation of generic repetition," as can be heard in Mayfield's adherence to established soul and funk generic conventions.\autocite[][137]{jameson_reification_1979}
As Rose writes, however, ``positioning repetition in late capitalist markets as a consequence of that market, marginalizes or erases alternative uses of and relationships to repetition that might suggest collective resistance to that system."\autocite[][72]{rose_black_1994}
Matthew P. Brown concurs in his exploration of funk music, claiming that its repetition ``engages in a critique of progress as it inheres in Western music."\autocite[By ``Western music," Brown refers to European-derived Western Art Music.][498]{brown_funk_1994}
These discussions of repetition is important as it helps to understand the presence of political ideology and national identity in Mayfield's soundtrack.

As McClary writes, musical progression may suggest individual, subjective growth.
Consistent repetition of musemes may instead suggest an aversion to personal innovation, instead prioritising collectivism.
Repetition can thus become symbolic of collective progression, as opposed to a more selfish, individualist aspiration.
Black popular music therefore implicitly critiques the Turneresque ideal of constant expansion and individualism at the expense of one's own community.

His use of musemes throughout his songs also therefore suggests a critique of a wholesale acceptance of capitalism.
However, they do so while working \textit{within} the economic structure they are critiquing, following generic conventions established through the mass reproduction of popular culture.
We can therefore understand the repetition found in Mayfield's soundtrack, and in black popular musics, as a form of critical nationalism, simultaneously challenging Turner's prioritisation of capitalist ingenuity while also following the practices that became established through that capitalist structure.
As I will detail below, this conflicted perspective on capitalism and entrepreneurialism is heard throughout many of \textit{Super Fly}'s songs.
But before I discuss these individual songs, I turn now to Mayfield himself in order to explore how Turner's capitalist ideology overlaps with Mayfield's.


% Adorno, perhaps most contentiously, asserted that repetition had the ultimate effect of ``seducing listeners into passive acceptance of the most barbarous elements of encroaching totalitarianism,” as surmised by Susan McClary.\autocite[][458]{mcclary_rap_2017}
% Tricia Rose explains how these perspectives equate repetition ``with industrial standardization and represents a move toward a single totalitarian code."\autocite[][71]{rose_black_1994}
% She pushes back, however, writing that perceiving repetition solely as a result of mass culture ``marginalizes or erases alternative uses of and relationships to repetition that might suggest collective resistance to" late capitalist structures.\autocite[][72]{rose_black_1994}
% Thus, for Rose, repetition in black popular music is a means to challenge ideological structures and mass commodification, albeit done within, and using the tools of, mass culture.

% These contrasting types should therefore not be understood as mutually exclusive, but Middleton writes that ``European popular music (before Afro-American influence) mostly use discursive repetition."


% Viewing repetition in black popular music as a binary opposition to the supposed linearity of Western Art Music seeks to reinforce the homogeneity of Turneresque Americanism, as well as its superiority by implying that the repetition heard in African-influenced popular musics is inherently ```childish' [and] `primitive.'"\autocite[][268]{middleton_studying_1990}


\subsection{Curtis's Capitalism}


While the generic conventions of \textit{Super Fly}'s score seems to challenge Turner's celebration of capitalism, Mayfield himself seems to represent a more favourable attitude to it.
I have noted above how Mayfield established his own record label and claimed full ownership and control over his music.
His business supremacy–in large part possible through his commercial popularity–therefore afforded him autonomy in an industry where artists were routinely exploited and undermined.
This exploitation extended to film music composers who were often viewed as dutiful employees expected to comply with their director's wishes, an issue I will explore in my later chapter on \textit{Alien}.
Mayfield, however, positioned himself as more than just a musician for hire, but as a ``business mogul."\autocite[][]{atria_curtis_2020}
This afforded him more freedom in his compositions for the films he scored and, in the case of \textit{Super Fly}, allowed him to compose several songs that explicitly condemned the actions of the film's protagonist.
That he was able to condemn much of the film's content demonstrates how valuable it was for the filmmakers to have a popular music artist for their score, and Mayfield leveraged this to vocalise his own ideology in seemingly direct contrast to the film's.

The influence and power that Mayfield had evidently accrued to this point, therefore stands as a highly laudable example of capitalist ingenuity.
Further, to Todd Mayfield's point cited above, Mayfield's success at doing ``what a black man in America wasn't supposed to do," exemplifies what Booker T. Washington claimed was essential in the fight for racial equality.\autocite[][141]{mayfield_traveling_2017}
Summarised by the National Business League, an organisation founded by Washington in 1900 to advance economic prospects for black individuals and businesses, Washington ``believed that economic progress would serve as a catalyst for broader societal change and improvement. He emphasized the need for African Americans to establish a strong economic network that would empower and advance the community."\autocite[][]{noauthor_legacy_nodate}
However, the image of Mayfield as an auteurial ``\textit{ubermensch}" is disrupted by his long-time collaborator and arranger Johnny Pate.\autocite[][67]{neal_what_1999}

Pate began working with Mayfield in 1963 and produced and arranged many songs for the Impressions.
His contributions to Mayfield's music at this time led Todd Mayfield to cite him as ``the man who had done more for [Mayfield's] music than anyone."\autocite[][213]{mayfield_traveling_2017}
After Mayfield went solo, they did not collaborate again until Mayfield began work on \textit{Super Fly} with Pate adding orchestral arrangements to Mayfield's otherwise sparse songs.
While this collaboration proved fruitful, a disagreement over the credits of two instrumental pieces–``Junkie Chase" and ``Think"–led to a breakdown of their personal and professional relationship.
Pate claimed to have written much of these songs and insisted on receiving a co-writing credit on these songs but Mayfield refused his requests.
Todd Mayfield described this as typical of his father's business practice: ``Curtis the businessman didn't share credit ... sharing credit would have meant sharing revenue."\autocite[][218-219]{mayfield_traveling_2017}
Mayfield's version of events is reflected in the film's opening credits which cite the music as ``composed, arranged and orchestrated by Curtis Mayfield," while Pate's role is credited solely as conducting (00:07:58; 00:08:07).

Pate spoke openly about his dissatisfaction, and the issue was discussed in trade magazines.
\textit{Billboard} magazine, for example, published an article which cited the refusal to grant Pate credit as tantamount to exploitation.
The article notes that ``many arrangers like [Pate] are dissatisfied with being the `invisible man,' so to speak, when `due credit' is being passed out for who does what when it gets down to the nitty-gritty of scoring a film."\autocite[][]{robinson_soul_1973}
In this regard, Mayfield's authorial control, financial reward, and artistic reputation came at the expense of Pate's who was viewed instead as an employee hired to assist in Mayfield's vision.
A statement made by Mayfield's lawyer made this explicit after Mayfield filed a defamation lawsuit against Pate: ``We aren't denying that Johnny Pate performed a very useful service in the arranging of the songs ... but he was an author for hire; he was paid a fee for his service."\autocite[][]{noauthor_mayfield_1972}
This narrative portrays Mayfield as the soundtrack's individual auteur, composing the music with minimal input from others, and subsequently rewarded as such.
The production of the soundtrack, then, contrasts the apparent prioritisation of collectivity that I argued the use of repetition references.
This reflects a common critique of the film's narrative, summarised by Eithne Quinn, that its ``celebration of black entrepreneurial individualism served to undermine communal action."\autocite[][163]{quinn_piece_2020}


Despite ostensibly exploiting his collaborator and prioritising his own entrepreneurial successes, throughout his career Mayfield demonstrated his desire to inspire communal action and progression.
This was detailed above in his political songs with the Impressions.
However, his insistence on asserting himself as the sole composer of \textit{Super Fly}'s soundtrack reveals his capitalist instincts which inherently required an individualist ideology.
His efforts to address racial inequality therefore contrast with his ideological alignment with the Turneresque American attitude of capitalist ingenuity and self-sufficiency.
And while he openly endeavoured to assert his capitalist dominance, the lyrics heard throughout many of \textit{Super Fly}'s songs demonstrate a clear repudiation of selfish, materialistic capitalism.
This capitalist critique is heard from the film's opening song,  ``Little Child, Runnin' Wild."


\subsection{The Songs}

\subsubsection{``Little Child, Runnin' Wild"}

This first song begins immediately as the camera looks down at a Harlem street.
After several seconds, the camera zooms and pans to follow a man on the street.
He approaches another man and we hear them argue about money which we soon learn is for purchasing drugs.
These men are later derogatorily referred to as the ``junkies."
They walk down the street and the camera cuts to a street-level shot of them as Mayfield's vocals are introduced (Figure \ref{fig:super-intro}).
They eventually enter a building and the camera zooms in on the darkened doorway before the next shot fades in and we see Priest laying in his bed.

\begin{figure}
    \centering
    \includegraphics[width=1\linewidth]{img/super-intro.pdf}
    \caption{\textit{Super Fly} opens with a shot of a Harlem street before the camera focuses on two men trying to buy drugs. The cut to the street-level shot is synchronised with the introduction of Mayfield's vocals}
    \label{fig:super-intro}
\end{figure}


The song is slighted edited from the version heard on the record, with some verses removed.
The first three verses follow an AABA structure, with A and B sections accompanied by an F\sharp minor and C\sharp minor, respectively.
As the camera follows the two men, it is suggested that they are each the titular ghetto child.
The verses are sung from the third person and detail their life experiences, while the third verse reveals the resentment that grows from this economic inequality.
The chorus then shifts perspective to the first person and laments the injustice of being born into a scenario where social mobility appears impossible (Table \ref{tab:super-little-lyrics}).
The song remains in the first person as two more verses detail the men's addiction and their need for drugs to salve their physical pains.
When the camera cuts to Priest, the chorus section is heard again though with different lyrics.
\begin{table}
    \centering
    \begin{tabular}{cc}\toprule
         Section& Lyrics\\\midrule
         Verse 1& Little child
\\
         & Runnin' wild
\\
         & Watch a while
\\
         & You'll see he never smiles
\\
         & 
\\
         Verse 2& Broken home
\\
         & Father gone
\\
         & Mama tired
\\
         & So he's all alone
\\
 &\\
 Verse 3&Kind of sad
\\
 &Kind of mad
\\
 &Ghetto child
\\
 &Thinkin' he's been had
\\
 &\\
 Chorus 1&I didn't have to be here\\
 &You didn't have to love for me
\\
 &While I was just a nothing child
\\
 &Why couldn't they just let me be
\\
    \end{tabular}
    \caption{The opening three verses and first chorus of ``Little Child, Runnin' Wild."}
    \label{tab:super-little-lyrics}
\end{table}

The choruses remain on the tonic F\sharp minor and deny any sense of structural progression or tonal development.
While repetition may be understood as a celebration of collective action, here the repetition and lack of progression reflects the brutal cycle of poverty that these men are trapped in.
The final line of the first of these choruses make this reading explicit: ``all my life has been this way."

To this point, the lyrics only vaguely reference the circumstances causing these men's dire socio-economic situation.
The song's final lines, however, reference ``the pusherman" as a key cause to their troubles.
He is portrayed as inhumane and solely fixated on his money, despite the human toll that it takes: ``can't reason with the pusherman, finance is all that he understands."
Mayfield here suggests that the pusherman is the one responsible for keeping the men in this cycle of poverty and drugs, implying that they are in his debt while also relying on the products he provides.
It simultaneously portrays the pusherman as a cold-hearted individualist whose empathy is secondary to his desire to accrue financial supremacy.
From the film's opening, Priest is thus aligned with the individualist capitalist, focused solely on his own wealth and unconcerned with the wellbeing of those around him.

The song returns three more times after this opening sequence, though each time without any vocals, edited to repeat the instrumental passages.
The full, unedited song as it appears on the album is never heard in the film.
As such, the song's most pointedly political verses are omitted:
\begin{quote}
One room shack

On the alley back

Control, I'm told

From across the track

Where is the mayor

Who'll make all things fair?

He lives outside

Our polluted air
\end{quote}
It is not clear why these lines were not included in the film but they make the song's critique of inequality far more overt and suggest that the socio-economic situation in Harlem is exploited by an unseen, upper class of society.
% It then specifically cites the mayor as one who could improve the lives of many but remains a distant and uninterested figure.
% He is physically removed from the poverty and environmentally hazardous situation depicted in the film and therefore positioned as above the people of Harlem.

The song's political critique is severely diluted without these lyrics.
It instead makes reference to the inequality that these men were born into but, without these accusatory verses, it does not reference the capitalist structures that allow this inequality to spread.
Nor does it cite the figures that could potentially improve the socio-political situation.
The music, meanwhile, exemplifies Middleton's musematic repetition.
As I argued above, this can be understood as a tacit critique of mass industrialisation and capitalist markets, and this appears to be the case in ``Little Child."
Such a reading remains only implied without Mayfield's more directly political verses which would have made the song's statement on class, political negligence, and individual autonomy explicit.

Figure \ref{fig:super-litle-verse} depicts the vocal's melody line, and motifs performed by strings, electric guitar and bass guitar.
These motifs are repeated with little variation throughout the verses.
Note that each of these verses end with a saxophone response to Mayfield's vocal line, though these are not transcribed.

\begin{figure}
    \centering
    \includegraphics[width=0.75\linewidth]{img/super-little-verse.pdf}
    \label{fig:super-litle-verse}
    \caption{Verse 1-3 of ``Little Child Runnin' Wild." The repeated musemes are highlighted and colour-coded. Dotted lines indicate the variation of these musemes as they transpose up a perfect fourth following the verses' AABA structure.}
\end{figure}


While the song's political commentary is not heard as explicitly in the film, Mayfield's critique of drug dealing as immoral and damaging to society and the individual's psyche is heard in most other songs, most obviously in the song ``Pusherman."


\subsubsection{``Pusherman"}

% Curtis playing, promoting him and the record
% The song is introduced immediately after tha frantic Junkie Chase as Freddie was getting away from robbing the guy. PM is more mellow, chill, cool. Suggesting the different dynamics between the two men, that inherently comes with the money and power.

Though heard only twice, ``Pusherman" is played for the longest duration of any songs on the soundtrack.
These two sequences are perhaps the most discussed of the entire film.
The first of these sequences follows a scene where Freddie and an accomplice rob a rival drug dealer.
Their escape is accompanied by ``Junkie Chase" (00:23:50), an instrumental piece featuring frantic, syncopated horns that create a sense of chaos as Freddie speeds away.
``Pusherman" enters almost immediately after ``Junkie Chase" finishes, in sync with a cut to a front-on shot of Priest's car as he drives through the city at night (00:24:01).
The song opens with a smooth bassline which reflects Priest's sense of comfort and confidence and sharply contrasts the frenetic energy of the previous cue and Freddie's panic as he flees the scene of the crime.
At this point, the power dynamic between Freddie and Priest has been well-established and is demonstrated by the difference in energy between these two pieces:
Freddie, the subjugated and desperate employee, is accompanied by stressful, chaotic music; Priest, the comfortable and financially secure boss, is accompanied by a laid back song that reflects the complacency that his money affords him.

The camera continues to follow Priest's car as Mayfield's vocals enter for the first verse in which Mayfield outlines the various roles the ``pusherman" performs for his customers: ``I'm your mama, I'm your daddy ... I'm your doctor when in need."
A guitar solo then enters, repeating the vocal melody line as octaves.
During this passage, the scene cuts to the interior of a bar where we see Priest enter, greeting other patrons on the way to his table.
The camera cuts again to show Curtis Mayfield himself performing the song with his band.
A medium wide shot then shows Priest listening to the song (Figure \ref{fig:super-curtis-live}).
Reaction shots of the audience show the crowd clearly enjoying the song as they groove in their seats, and the band get a rapturous applause when they finish their song.

\begin{figure}
    \centering
    \includegraphics[width=1\linewidth]{img/super-curtis-live.pdf}
    \caption{Priest watches Curtis Mayfield performing ``Pusherman"}
    \label{fig:super-curtis-live}
\end{figure}

The film's narrative is essentially paused while Mayfield performs the song almost in its entirety.
This sequence therefore essentially acts as an advert for the soundtrack.
As I discussed above, the music was key to the film's success and a primary point of focus for its marketing campaign with promotional material advertising that audiences would get to ``see and hear CURTIS MAYFIELD play his Super Fly score!" (Figure \ref{fig:super-lobby-card}).
While this is thus a clear attempt to attract larger audiences, the choice of song he performed is a curious one as it was not one of the songs released as a single and it sees Mayfield directly critique the character of Priest.
Mayfield's lyrics nevertheless present the pusherman as a cool, aspirational figure.
This is foregrounded in the second verse where Mayfield's pusherman brags of his style, virility, and masculinity:
\begin{quote}
Ain't I clean? Bad machine

Super cool, super mean

Dealing good, for the Man

Superfly, here I stand

Secret stash, heavy bread

Baddest bitches in the bed

I'm your pusherman
\end{quote}
Even when boasting of his capitalist success, though, the pusherman admits to his role as, essentially, a stooge for ``the Man."
As such, while his financial success affords him materialist gains, his individual autonomy is curtailed as he admits to dealing for those even higher than himself.
We learn later in the film that the Man in this situation is a high-ranking police officer who represents the structures that maintain racial and economic disparity in contemporary society.
Priest thus works for the very socio-economic system that reinforces his own inequality but is ultimately content with this dynamic as long as it affords him the lifestyle he desires.

The bridge complicates this when the pusherman describes himself as ``a man of odd circumstance, a victim of ghetto demands."
These lyrics suggest a sense of inevitability as the demand for drugs in his neighbourhood explain why he began dealing drugs in the first place and that this was his only chance to achieve financial success.
Priest's partner Eddie specifically states this conundrum earlier in the film when he admits that dealing drugs is ``a rotten game" but is the ``only one the Man left us to play" (00:19:07).
For Eddie, the lack of personal freedom and the immorality of their profession is worth it for their luxurious lifestyle.
He states this when asking why Priest wants to get out of the business: ``you're gonna give all this up? Eight-Track Stereo, colour TV in every room, and can snort a half a piece of dope everyday? That's the American Dream" (00:18:21).
Mayfield's lyrics likewise celebrate this materialism in the third verse, where he brags of ``making money all the time."
Eddie and Mayfield's music therefore highlight the challenges and lack of individual autonomy for those born into these socio-economic circumstances, and although they lament this situation they justify it as long as they are granted their financial success.

For Turner, this may well be a deeply troubling situation as he called for the self-sufficiency of the American character which, he believed, should come hand in hand with a dominant capitalist acumen.
The socio-economic structures at play, however, allow Priest and Eddie to demonstrate their capitalist ingenuity only when sufficiently subjugating themselves to the Man.
They celebrate the ``American dream," but acknowledge the limits placed on their autonomy.
Eddie again references this later in the film when he and Priest discuss the Man's power over them and he says he is fine sacrificing his individualism for financial gain: ``I'm glad [the Man's] using me. Because I'm gonna make a piss pot full of money and I'm gonna live like a prince" (01:17:51).

Mayfield's lyrics, along with Eddie's comments, therefore suggest an embrace of the Turneresque American ideal, as they aspire to capitalist dominance and a life of material luxury.
They simultaneously critique the contemporary reality that makes \textit{becoming} the Turneresque figure impossible for many.
Priest, of course, attempts to break free from the structures keeping him in his place, stop working for the Man, and assert his autonomy.
He therefore clearly aspires to become the figure that Turner outlined, regardless of the warnings and critiques outlined in Mayfield's lyrics.


\subsubsection{Photo montage}

The second time ``Pusherman" is heard comes in the aforementioned photo montage where Priest's dealers are seen selling to a variety of different customers.
It gradually fades up as Priest and Eddie agree to move forward with selling the large amount of cocaine they have bought in order to help Priest retire (00:57:10).
The song is played almost in full; however, it is edited to rearrange the order of the verses.

The first verse plays as it appears on the album version.
These lyrics are accompanied by still images of three men conversing and laughing in the street, associating them with the various roles that the lyrics claim the pusherman performs: mother, father, doctor, and friend.
The bridge is heard next, moved earlier in the arrangement.
It plays as one of the men is shown weighing and bagging the cocaine.
Unlike Priest's other associates, this man is wearing a suit which seems to give his criminal actions a professionalism.
His attire draws a parallel between this criminal underworld and the legal capitalist world and, as William L. Van Deburg notes, exemplifies blaxploitation convention wherein hustlers are presented as ``aspiring magnates of materialism."\autocite[][140]{van_deburg_black_1997}
Van Deburg adds that these figures ``took considerable pride in the corporate structures and complex distribution networks they created," and this image of a seemingly white-collar businessman conducting his business in a controlled and precise manner demonstrates this clearly.\autocite[][140]{van_deburg_black_1997}
Furthermore, the difference in his appearance, compared to that of Priest's and his other dealers, reinforces the lack of opportunities afforded to those born in such circumstances whether they fully embrace the hustler lifestyle or a more legitimate image of professionalism.
This is reflected in the accompanying lyrics which portray the dealers as ``victims" of their surroundings.
Two different men are then also shown measuring out the cocaine.
In an apparent critique of the American dream cited by Eddie earlier in the film, they are seen using a board with the United States flag to weigh the drugs (Figure \ref{fig:super-montage-start}, bottom right).

\begin{figure}
    \centering
    \includegraphics[width=1\linewidth]{img/super-montage-start.pdf}
    \caption{A montage depicts Priest's employees measuring and bagging the cocaine.}
    \label{fig:super-montage-start}
\end{figure}

The rest of the montage sees Priest's dealers distributing the cocaine to clientele from all walks of life in a demonstration of the size of Priest's empire and the reliance that so many people have on a subculture that many would perhaps pretend does not exist.
Despite the size of his operation, Priest acknowledges this latter point when trying to convince Eddie to retire with him, claiming that the Man does not even consider them ``real," simply a means to further increase profitability (01:17:38).

Throughout the montage, each of Priest's customers are seen taking the cocaine, smiling and enjoying the experience.
Such euphoric reactions match the song's boastful lyrics and celebrate not just Priest's ingenuity but the service he is providing.
This celebration is diminished at one point when the lyrics reveal Priest's most personal insecurities:
\begin{quote}
Got a woman I love desperately

Wanna give her something better than me

Been told I can't be nothing else

Just a hustler in spite of myself

I know I can break it

This life just don't make it

\end{quote}
These lyrics mirror Priest's quest to quit dealing as well as conversations wherein characters lament the lack of opportunities to be anything other than a hustler.
The accompanying images over these lines depict a man and a woman taking cocaine and embracing before he walks away from her, possibly to allow her to find ``something better than me" (Figure \ref{fig:super-montage-better}).
The self-reflection does not last, though, as the first verse is repeated and the pusherman reasserts himself as a key societal figure.
The song continues as the montage ends and the camera cuts to Priest and Eddie on the street walking into a bar.
They sit at a table and the song fades out.
\begin{figure}
    \centering
    \includegraphics[width=1\linewidth]{img/super-montage-better.pdf}
    \caption{A man and woman shown during \textit{Super Fly}'s photo montage. These images are accompanied by Mayfield's lyrics depicting the pusherman's relationship.}
    \label{fig:super-montage-better}
\end{figure}



This sequence strongly critiques the pusherman, the societal factors that force people into this role, and the personal toll it can take.
In this way, the song's use of musematic repetition–its repeated bass motif and electric guitar chords–can be understood as an example of the kind of capitalist critique that I discussed earlier.
As detailed above, repetition is capable of critiquing commodified mass culture while functioning within it.
This seemingly contradictory approach is visualised in Priest's associate wearing his suit, thereby embodying the imagery of the corporate culture that he is essentially working in contrast to. 

``Pusherman's" lyrics similarly critique the socio-economic situations that resulted from this capitalist market (``a victim of ghetto demands"; ``been told I can't be nothing else") while also celebrating the materialist gains that the market makes possible (making money all the time"; ``my El-D and just me").\footnote{``El-D" refers to Priest's car, a Cadillac Eldoraldo }
The bassline, which repeats with almost no variation until the bridge, also refers to the circularity of this situation and references the cycle of poverty and crime which is one of the soundtrack's and the film's key themes.
This circularity is disrupted in the bridges when the bassline introduces a new motif.
Significantly, it is in these sections that the lyrics become far more introspective and aware of the negative aspects of this lifestyle, whereas the verses seem more positive.
The coupling of this self-awareness with a new bassline suggests an attempt to break free from the pusherman's illegal, immoral, and damaging lifestyle.
However, as Brown claims is typical of the funk genre, the song inevitably reverts to the original bassline and thus denies the possibility for harmonic or melodic progress, much as the capitalist structures in place deny Priest any legitimate attempts at upwards mobility.

For many, the obvious enjoyment that Priest's customers find in using cocaine, the self-mythologising lyrics, and the stylised presentation of the montage contribute to an overall celebration of Priest's capitalist endeavour despite its blatant criminality.
For example, Ron O'Neal, who defended the film against criticism upon its release, complained in a 1999 interview that this sequence ``looks like the greatest cocaine commercial the world's ever seen."\autocite[Ron O'Neal, quoted in][16]{ryfle_politics_2019}
The song's critique of this lifestyle is therefore complicated, the damaging effects of the hustler lifestyle on one's self-perception (``wanna give her something better than me") juxtaposed with the depiction of cocaine usage as glamorous and enjoyable.

Despite these more pessimistic details, this sequence arguably appears more like a celebration of Priest's personal wealth and individualist self-sufficiency.
This comes at the expense of Priest's community, which he exploits by profiting from the drug problem.
This theme is exacerbated in the following scene which sees Priest making his personal ideology about his own self-interest explicit.
As noted, the montage ends as it cuts to Priest and Eddie entering a bar.
The song continues without any lyrics as they sit at a table and are approached by three men.\footnote{The end credits cite these unnamed characters as ``militants." I will therefore refer to them as such.}
It abruptly stops as the militants stand over Priest's table trying to convince him to contribute to their cause, ``building a new nation for black people."
They chastise Priest, accusing him of exploiting black people and encouraging him to help pay back to the community: ``black folks been mighty good to you. And you owe those people something too ... It's time for you to start paying some dues."
Priest, however, refuses to contribute.

Though this scene does not feature any music, it is notable that it comes immediately after a sequence in which we saw the full extent of Priest's successful business.
In the montage, Priest's empire seems commendable and a result of his admirable capabilities.
His meeting with the militants, however, implies that Priest's actions are a net negative to his community and his refusal to help these men's cause demonstrates his lack of interest in sharing his wealth for the benefit of others.
He offers to help only when their attempted revolution turns violent.

Similar to the montage sequence, this scene has provoked criticism as Priest implies that attempts to enact change through peaceful protest and community building are ineffectual.
For William Lyne, this rebuttal shows a ``dismissal of Black Power politics," with Priest demonstrating his dominant masculinity and ridiculing the militants' ideology: ``as they leave with their tails between their legs, the `militants' have not only bowed to Priest's superior masculinity, they have also relinquished any claims on effective resistance."\autocite[][43]{lyne_no_2000}

With these consecutive sequences, Priest has demonstrated his capitalist supremacy and rugged, individualist dominance over other men.
He is therefore aligned with the assertive, masculine American ideal seen in \textit{Big Jake}.
These themes are reflected in ``Pusherman," with its largely boastful lyrics; yet, its use of introspection reveals awareness of the downsides of these Turneresque traits.
We are therefore presented with an embrace of Turner's thesis and an acknowledgement of its limitations and negative consequences.
This latter point is also heard explicitly in two other songs, ``Eddie You Should Know Better," and ``Freddie's Dead."




\subsubsection{Eddie}

% ``Freddie" only ever occurs as an instrumental arrangement in the film. Apparently Curtis had not written the lyrics yet. But it was the first single and was successful so still worth discussing in here. Single and album both released before film so a big spoiler. The song even plays before we meet Freddie.

``Eddie You Should Know Better" opens after a scene where Priest and Eddie visit Scatter, their former mentor and a retired drug dealer, in the back room of his bar.
Priest implores Scatter to help them acquire enough cocaine for them to retire on but he is reluctant to help them.
During this exchange Eddie insults Scatter and he pulls a gun on him.
Priest is able to diffuse the situation and finally convinces Scatter to help.
Priest and Eddie leave and discuss the agreement they just made and Priest scolds Eddie for how he behaved, warning him that Scatter might have killed him.
As they walk back into the bar, an instrumental version ``Eddie You Should Know Better" begins to play.

Lyrically, the song is sung from the third person perspective and addresses Eddie directly, opening with the title line and chastising him for his selfish actions:
\begin{quote}
    Eddie, you should know better
    
    Brother, you know you're wrong
    
    Think of the tears and the fears
    
    You bring to your folks back home
    
    They'd say, ``where did he go wrong, my Lord?"
\end{quote}
With the song playing directly after his altercation with Scatter, albeit as an instrumental version, these opening lines seem to refer to the way he behaved in their meeting and reflect Priest's warning to Eddie that Scatter could have killed him.
The lyrics chastise Eddie for more than this infraction though, discussing the shame and worry that he has caused his family.
Eddie's apparent uninterest in the pain he may be causing his family points to his selfishness and focus on his pursuing financial success.
This is made more explicit in the second verse where Eddie is portrayed as ``everybody's friend," but only when he stands to gain from the relationship:
\begin{quote}
    The only time he'll choose you
    
    When there's something to lose through
    
    His personal loss
    
    And the friend pays the cost
\end{quote}
Eddie's selfishness and willingness to make friends ``pay the cost" foreshadows the film's ending as he later betrays Priest to prevent him from leaving with the money.
It also foregrounds the conflict between individualism and collectivism that is a key theme in Mayfield's soundtrack and much of his wider catalogue, black popular music, and Turner's thesis.
Mayfield's evident disapproval of this self-centred attitude is contrasted with how much of the film seems to idolise Priest, even though he shows similar selfishness and willingness to do whatever it takes to get what he feels he deserves. 

Further contradiction is heard in the song's final lines as Mayfield laments that Eddie's greed has made him ``so very blind."
This is clearly a critique of Eddie's materialism and desire for power and money which has him to lose sight of important things such as friends, family, and community.
The last line suggests that this may lead to Eddie's downfall: ``I don't think he's gonna make it this time."
As Mayfield's lyrics imply, without respect and connection to his community Eddie will be unable to survive.

Musically, ``Eddie" represents a significant difference from most of the soundtrack's other songs which which typically alternated between two chords, or simply remained on the tonic chord.
As Johnny Pate notes, however, with ``Eddie" ``you've got chord structure, you've got beautiful chord changes, plus a great melody."\autocite[Johnny Pate, quoted in][213]{mayfield_traveling_2017}
This provided Pate with a greater challenge in arranging and orchestrating the song, while the use of extended chords add a constant sense of movement as the song never rests on any fully consonant chord for more than a few bars.
Figure \ref{fig:super-eddie-leadsheet} shows the song's chord structure and vocal melody line, and makes evident the relative linearity that is not found in Mayfield's other \textit{Super Fly} songs.
Though the music is repeated, the second verse features different lyrics and a varied melody line.
This thus appears an example of Middleton's discursive repetition.


\begin{figure}
    \centering
    \includegraphics[width=1\linewidth]{img/super-eddie-leadsheet.pdf}
    \caption{The chord structure and vocal melody line from ``Eddie You Should Know Better."}
    \label{fig:super-eddie-leadsheet}
\end{figure}
The difference in structure from other songs on the soundtrack reflects Eddie's own ideology.
As I argued above, musematic repetition within black popular music can be seen as a means to celebrate a sense of community and collectivism.
The lack of such repetition in ``Eddie" connotes a lack of collectivist ideology.
This reflects Eddie's aspiration to secure his own financial situation and attain enough personal wealth to ``live like a prince" (01:17:51). 
The song's minor tonality and the lyrics' lament that Eddie ``should know better" suggest a negative perspective on his prioritisation of himself.
The circularity of the song, as the structure is essentially repeated after being played once, reflects Eddie's inability and his uninterest in breaking out of this cycle.

Mayfield's chord progression creates a slightly uneasy atmosphere.
Indeed, while the chords generally remain in the tonality of F\sharp minor, there are brief instances of dissonance from extended and non-diatonic chords.
These instances are heard through Mayfield's use of dissonant extended chords, most notably the consecutive C\sharp7(\sharp9) and C\sharp7(\flat9).
Such dissonances could be understood as critiquing the the hustler lifestyle discussed within the lyrics.
However, this perspective reveals an over-reliance on Western classical music theory and, as I have suggested above, Western musical traditions do not necessarily pertain to black popular music traditions.
It would therefore be disingenuous to draw such conclusions.
Rather, the music is functioning within the established conventions of its genre by employing extended chords and deviating from Western diatonic harmony.
We can thus see a comparison between ``Eddie" and its namesake, as both adhere to accepted behaviours of their respective communities: ``Eddie" rejects academicised tonal practice and utilises dissonance to create its soundworld in a manner typical of black popular music; Eddie rejects the rules and morality imposed by ideological state apparatuses that do not work with his criminal surroundings.

``Eddie's" chordal progression nevertheless demonstrates a greater alignment to Western Art Music than the rest of Mayfield's soundtrack, as, following Brown, it eschews musematic repetition in favour of more ``harmonic, tonal [and] melodic development," a ``trajectory [that] is de-emphasized in the African-American tradition"\autocite[][497]{brown_funk_1994}
This deviation from black popular music traditions suggests a rejection of its ideology, which we understand as prioritising collectivism.
This reflects Eddie's own selfishness and materialist aspirations as he similarly turns his back on his community.

Both ``Eddie" and Eddie therefore reflect a prioritisation of personal growth and individualism, and the focus on capitalism as a means to assert one's dominance.
As such, we can hear in this song an example of Eddie's adherence to the characteristics endorsed by Turner.
Mayfield seems to tell us that this is not an entirely commendable figure.

It is also important to note that Turner's American figure is essentially only idolised here because it is portrayed as the characters' only choice: either they embody a Turneresque identity and attain financial security and societal dominance; or reject it and risk being taken advantage of or succumbing to the vices pushed by the likes of Eddie and Priest.

With this understanding ``Eddie" denounces the pursuit of capitalist wealth at the expense of one's community and by extension rejects Turner's endorsement individualism over collectivism.
However, the necessity to follow the lessons of Turner's thesis remains implied. 
This therefore retains the same complex rebuke of Turner's thesis that is heard in other songs by avoiding a simplistic endorsement or rejection.


\subsubsection{Freddie}

``Freddie's Dead" offers a similar critique of the hustler lifestyle.
Although it is only heard in the film as an instrumental version, it was released as the lead single from the soundtrack album prior to the film's release, meaning its lyrical may well have been familiar to many audiences.\footnote{``Freddie's Dead" was released in July 1972, shortly before the film was released on 4 August 1972. The song entered the Billboard charts on 19 August before peaking at number four on 4 November.}
The lyrics focus on the fate of Priest's employee Freddie, who mugs a rival drug dealer and is interrogated by the police before he attempts to flee and is killed by a passing car.

Though the film's narrative makes that Freddie's is killed because of his own actions, ``Freddie's Dead" focuses on the societal environment that he was born into and which limited his opportunities, leaving him with little other choice than to engage with the criminal underworld.
Brad Schreiber discusses the song's meaning writing that it ``emphasized that institutional poverty always led to literal and figurative dead ends for those who could not resist the lure of dealing.”\autocite[][106]{schreiber_music_2020}

The song opens with an electric and bass guitar riff played in unison.
A brief time change to \timesig{2}{4} and a drum fill precedes the first verse where the main museme is introduced, performed by guitar, bass, and flute
The vocal line also largely follows this museme, but does so slight variations.
Figure \ref{fig:super-freddie-riff} shows a reduction of the score, highlighting the song's reliance on this particular motif as well as its repetition of the tonic chord of C\sharp minor, albeit with extended variations such as added sevenths and thirteenths.

\begin{figure}
    \centering
    \includegraphics[width=1\linewidth]{img/super-freddie-riff-port.pdf}
    \caption{The first verse of ``Freddie's Dead," showcasing the importance of the primary ascending museme.}
    \label{fig:super-freddie-riff}
\end{figure}

The only variation from this comes in the chorus which sees a progression to the subdominant of F\sharp minor and a more varied bass line which ascends in a syncopated pattern before descending.
The song's lamenting of Freddie's fate is most clearly laid out in the second chorus:
\begin{quote}
Everybody's misused him
    
Ripped him up and abused him

Another junkie plan

Pushing dope for the man

A terrible blow but that's how it goes
\end{quote}
Freddie is here presented as a victim of his surroundings and of Priest as much as he is the victim of the driver who killed him.
The near-constant repetition of the primary motif depicted in Figure \ref{fig:super-freddie-riff}, hints at the apparent inevitability of Freddie's fate as well as the circularity of the poverty-stricken environment he was born into.
And yet, there is a brief instance of significant musical progression which threatens to upset this circularity.
At this moment, the riff is transposed up a semitone and continues to repeat in its new key of D\natural minor.

The semitone transposition, heard in ``Freddie's Dead" is an example of what Robert Bailey calls ``'expressive' use of tonality" in his examination of Wagner's \textit{Ring} cycle.\autocite[][51]{bailey_structure_1977}
Such transpositions are commonly found within popular music, film music, and Western Art Music.
Expressive tonality, for Bailey, is when a ``repetition or recall of a passage is transposed up to underscore intensification, or shifted down to indicate relaxation. These shifts are usually made by a semitone or a whole tone."\autocite[][51]{bailey_structure_1977}
In this case, the motif’s modulation functions to intensify the spectator’s apprehension and the sequence’s suspenseful atmosphere.
Frank Lehman further discusses expressive tonality, noting its ``ubiquity” in film music.\autocite[][55]{lehman_hollywood_2018}
He adds that ``standard shorthand for transposition is T\textsubscript{n}, where T stands for transposition and \textit{n} stands for the number of semitones by which the target sonority/key is shifted."\autocite[][54]{lehman_hollywood_2018}
Following this convention, the transposition of the ``Freddie's Dead" motif here is written as T\textsubscript{1}.
As the modulated motif involves each note being transposed by the exact same interval–``verbatim” in Lehman’s terms–it is understood as a ``\textit{mode-preserving}” transposition.\autocite[][54]{lehman_hollywood_2018}

This is the only instance of transposition throughout the soundtrack and therefore the only time that the music appears to progress beyond its chord structure.
Returning to McClary's summation of musical progression as emblematic of personal development, this instance of expressive tonality suggests an opportunity for Freddie, the song's subject, to break from the cycle of poverty and crime.
However, there are several points that tell us he will not.
First and foremost is the song's literal meaning: Freddie is dead and can therefore never transcend the limitations society has placed upon him.
Further to this point, as the song is heard before Freddie is even introduced, it is imbued with dramatic irony, with audience's aware of the song's title and lyrics being made told of Freddie's fate.
A second issue is the mode-preserving nature of the transposition which denies the melody any progression, despite its new tonal centre.
This suggests that, despite the self-improvement represented by the upward transposition, Freddie's situation and prospects ultimately remain the same.
A third issue is in the comparatively brief section in which the song remains in the new tonal centre of D\natural minor.
After six bars, the song transposes back down a semitone to the original key of C\sharp minor, representing Freddie's inability to break the cycle.

While I have discussed this song so far as it pertains to Freddie, it is important to note that it is not associated with him throughout the film.
Instead, it becomes more associated with Priest since it is used to introduce him and he is onscreen each time the song is played.
As I explained above, each of the song's occurrences in the film are completely instrumental, making it easier for a song about Freddie to become associated with Priest.

The song is first heard during the title sequence in which we follow Priest through the Harlem streets.
He is introduced in bed with a girlfriend and he leaves her in silence, ignoring her when she asks if he will be back soon.
A cut to an external shot of the girlfriend's building shows Priest leaving and ``Freddie's Dead" opens in synchronisation with this cut.
Priest crosses the road to his car and drives away.
Another cut after he has driven away moves the camera in front of the car.
High strings perform a descending slide as the title card seems to mirror this effect by slowly zooming towards the camera.
The title appears superimposed over Priest's car, making clear that he is the "super fly" figure described by the film's title.
The song progresses to its chorus seconds after the title card is at its largest, this time accompanied by a full string section providing an uplifting and celebratory introduction to Priest.

Priest continues his journey with the camera alternating between profile close-ups of his face and a front-on shots of his car (Figure \ref{fig:super-freddie-intro}).
As he cruises through the city streets the camera's low-angle shots of his car convey his power and idolise the car as emblematic of his financial status.

\begin{figure}
    \centering
    \includegraphics[width=1\linewidth]{img/super-freddie-intro.pdf}
    \caption{\textit{Super Fly}'s title sequence cuts between interior profile shots of Priest and exterior shots of his car while ``Freddie's Dead" plays.}
    \label{fig:super-freddie-intro}
\end{figure}

Richard Dyer writes that introducing protagonists in this way is typical of the blaxploitation genre and cites \textit{Super Fly}'s opening as one of many examples in which characters are first seen ``moving through the streets to the sound of contemporary funk."\autocite[][156]{dyer_space_2012}
For Dyer, these introductions serve to establish blaxploitation protagonists' ``connectedness to this world," and, by extension, their communities.\autocite[][156]{dyer_space_2012}
Priest is thus introduced as belonging to this environment, comfortably driving through the streets, and demonstrating his connection.

However, \textit{Super Fly} differs from other blaxploitation films which open with similar sequences as its camera focuses almost exclusively on Priest.
These other films–for example, \textcite{davis_cotton_1970}, \textcite{parks_shaft_1971}, and \textcite{cohen_black_1973}–introduce their characters as physically connected to their communities by showing them walking on foot through the streets, or they otherwise feature bustling streets with other members of the community.
Figure \ref{fig:super-blax-intros} shows stills from these films' title sequences to demonstrate the contrasting ways they introduce Harlem and its community.
Priest is introduced, therefore, in the same way as the villain from \textit{Cotton Comes to Harlem}, who is similarly shown within his own car driving through the streets of Harlem, within but removed from the community (see top row in Figure \ref{fig:super-blax-intros}).
In contrast to these figures, \textit{Super Fly}'s introduction solely lingers on Priest's face and car, and does not depict any other people from the neighbourhood.
Priest is shown as connected to, but simultaneously removed from, his Harlem neighbourhood.
Mayfield's score therefore becomes connected to Priest himself and not the space he inhabits, celebrating him as a lone figure separated from his environment.

\begin{figure}
    \centering
    \includegraphics[width=1\linewidth]{img/super-blax-intros.pdf}
    \caption{Images from other blaxploitation films that also introduce their principle characters as connected to their community.
    Top row: \textit{Cotton Comes to Harlem} shows its antagonist removed from the bustling streets while seemingly still connected.
    Middle row: \textit{Shaft} sees the title character walking through New York's streets.
    Bottom row: \textit{Black Caesar}'s Tommy Gibbs is introduced as an established fixture within the community as a shoeshiner.}
    \label{fig:super-blax-intros}
\end{figure}

The association between ``Freddie's Dead" and Priest continues throughout each of the song's occurrences.
The next two times it is heard, it accompanies Priest as he drives his car (00:13:37) and as he returns home (00:16:42).
Its final occurrence comes toward the film's end, when he returns to his apartment and speaks with Eddie who seems to have been waiting for him (01:16:49).
The scene opens with a close-up shot of Eddie's hands dropping the needle on a record.
Priest enters the room and Eddie greets him as he switches the record player on and ``Freddie's Dead" begins.
With the song emitting from within the film's diegesis–hinted by Eddie setting up the record player and the song's lower volume and sound quality–it is here unequivocally associated with Priest since he clearly owns the record.
Priest owning the record acts as a further advert for the physical soundtrack as something audiences could own, another example of the fimmakers attempting to boost the film's commercial appeal.

Eddie asks Priest what he ``thinks of Scatter OD'ing," referring to their business partner who seemingly accidentally overdosed.
Priest, however, knows he was killed by the corrupt police officers who Priest and Eddie now inadvertently work for.
After roughly six seconds, the ``Freddie's Dead" transposition occurs as Priest corrects Eddie's assumption: ``OD? Scatter didn't OD, somebody killed him!"
Returning to Bailey's assertion that ascending expressive tonality can increase intensity, the synchronisation between ``Freddie's" transposition and Priest's claim about Scatter heightens the tension as the film reaches its climax and it is made clear that Priest's life is in danger.

Again, knowledge of the omitted lyrics added an extra degree of meaning to this sequence, hinting that Priest may well meet the same fate as Freddie.
This becomes exacerbated as Priest implores Eddie to leave with him:
\begin{quote}
Priest: Let's take what we have now and split.

Eddie: Are you crazy? Now, you know damn well who killed Scatter. Now, we fuck up, he'll kill us too. Besides, we got it beat. We'll make a fortune. So don't rock the boat, man.

Priest: It's worse now than it was before. Look, that man owns us. You understand that, Eddie? To him we're not real. He'll just use us and then kill us.
\end{quote}
This recontextualises the second chorus of ``Freddie's Dead," wherein Mayfield laments that ``Everybody's misused him, ripped him up and abused him."
Whereas Freddie is the one named in the song and the sole identified victim of the Man, in this sequence the song's concerns become more universal with Priest and Eddie also becoming the ``misused" subjects who are ``pushing dope for the Man."
Eddie, however, is not concerned about this situation.
Rather, he claims he is happy to be used by the Man and to perpetuate the cycle of crime, drug abuse, and poverty, provided he can be wealthy.
This cycle is again reflected with ``Freddie's" repeated motif while the song's minor tonality mirrors Eddie's awareness of his lack of opportunities and how the system is rigged against him.

Their conversation comes to an end as Priest tells Eddie he does not trust him and is going to take the money and leave.
Despite the song clearly coming from the record player, the song ends without anyone switching off the stereo.
The song's abrupt ending underscores the drama of Priest and Eddie's falling out.




% Using black popular musics to introduce urban, inner-city environments is a common, and often problematic, filmic trope, and acts to portray these environments as modern and also as racially-coded.

% "awareness raising self criticism" typologies for black music.



% ------------------------------
% % TRYING TO SUMMARISE THE IMPORTANCE OF ALL THE REP. CHAT
% % In the case of black popular music , it is also openly AMERICAN (wears its African influence proudly, and endorses its capitalist background). We can
% % therefore understand it as a form of critical patriotism/nationalism: pushing against a full-throated endorsement of Turnerism, while still endorsing the national identity he supported and embracing the capitalism his Americanism necessitates.

% % Such context is important when thinking about Mayfield's score and Turner's American identity as it highlights a clear similarity between both approaches to national ideology.

% % Although some have examined repetition in black popular music as a means to differentiate it from European-derived musics–and by extension, reinforcing a binary understanding of national identity–the prevalence of repetition in both musics points to the inherent falsity of Turner's thesis since it highlights a shared cultural practice despite ostensibly deriving from differing cultural backgrounds.

% % While Turner seemed to claim that American culture was solely derived from its European ancestry, the shared practice of repetition in both European and African musical traditions points to the inherent falsity of his perspective.




% THIS IS ALMOST ENTIRELY REFUTED BY MCGRATH AND MORE RECENT WORK! Or is it... LACK OF RESOLUTION IN FUNK IS PART OF THE POINT(?) - PERSEVERENCE ETC.
% Contrast repetitive funk/soul music with Western's developing themes. 
% For Brown, “the Western tradition concentrates on harmonic, tonal or melodic development," and the lack of this development is often perceived as inherently displeasing. Funk music, on the other hand, “looks for no such resolution."(Brown 1994, 498) As such, “On and On’s" repeated E minor instead reflects tradition funk aesthetics and, by adhering to these traits, reiterates the genre’s theme of personal and political perseverance.
% (from Claudine - does he mean Western European or western genre? Either way useful contrast between euro/afro).


\subsection{Superfly}

The final song that must be consider is the title track.
It is not heard until near the end of the film, after the sequence in which Priest's mentor Scatter is killed and Priest is shown alone in his home seemingly formulating a plan to escape the situation he has gotten into with the corrupt police officers.
This scene does not feature any dialogue and shows Priest pacing alone, loading his gun, and eventually leaving after he seems to have decided on his next course of action.
The scene cuts to an exterior shot of a street, and the cut is synchronised with the introduction of ``Superfly."
The cue opens with a syncopated horn motif over a repeated bassline (Figure \ref{fig:super-superfly-intro}).
From here, the song continues as it does on the record with the vocal melody introduced shortly after.

\begin{figure}
    \centering
    \includegraphics[width=1\linewidth]{img/super-superfly-intro.pdf}
    \caption{The instrumental introduction that first plays during the first occurrence of ``Superfly."}
    \label{fig:super-superfly-intro}
\end{figure}

We see Priest park his car and enter a restaurant where he sits with two men while the camera remains outside watching him through the window.
As they talk, Priest hands two large envelopes to the men. 
They continue to talk before he gets up, shakes their hands, and leaves.
The camera lingers on the two men before cutting to a close-up of the hood ornament on Priest's car, synchronised with Mayfield's delivery of the title line, associating Priest with the ``super fly" figure being sung about.
The song ends very abruptly with a cut to Eddie putting on the record of ``Freddie," a sequence I discussed above.

During the meeting, the camera alternates between close up shots of Priest and the two men he is meeting, and a medium shot of all three of them (Figure \ref{fig:super-superfly-meeting}). 
Despite the camera clearly showing the meeting there is no diegetic sound.
Instead, ``Superfly" is the only thing on the soundtrack which serves to foreground the song's lyrical message.
The lyrics largely seem to praise Priest, focusing on his intelligence (``this cat of the slum, had a mind wasn't dumb") and his masculine dominance over others (``the man of the hour has an air of great power, the dudes have envied him for so long'').
And yet, despite idolising this inner-city version of Turner's masculine ideal, ``Superfly" is ultimately a warning song about the dangers of Priest's hubris and the hustler lifestyle.

This seems largely inspired by the film's narrative which, to this point, has led Priest to a highly dangerous situation as he works for a corrupt police officer with the power to have him killed.
This is emphasised in the previous scene where Scatter is killed, suggesting what could easily happen to Priest if he steps out of line. 
The chorus's final line addresses these high stakes: ``the only game you know is do or die."

\begin{figure}
    \centering
    \includegraphics[width=1\linewidth]{img/super-superfly-meeting.pdf}
    \caption{Enter Caption}
    \label{fig:super-superfly-meeting}
\end{figure}

With no dialogue, it is unclear what who these men are and why Priest is giving them envelopes.
But the lyrics, foregrounded by the lack of other sound, hint at him carrying out a plan that will save his life.
While Mayfield's chorus outlines the dangers facing Priest (``the game he plays, he plays for keeps"), the bridge carries this metaphor further.
This section of the song enters when Priest pulls out the first envelope.
The camera lingers on it while the lyrics outline his determination to come out on top:
\begin{quote}
The game he plays, he plays for keeps

Hustling times and ghetto streets

Trying to get over
\end{quote}
Although earlier verses and the chorus praises aspects of Priest's character, it largely remains generally pessimistic about his situation and the negative aspects of his life.
This section, however, suggests that he has a plan to help him win ``the game."


Somewhat similar to ``Eddie You Should Know Better," ``Superfly" features a wider range of chords than most songs on the soundtrack.
While its verses remain on the tonic of E7, the chorus introduces more variation, but it is not until the bridge that a large shift seems to occur.
This section opens with a shift to the \flat VI chord, a C major 7.
Figure \ref{fig:super-superfly-bridge} provides a reduction of this section with the chord progression and vocal melody.
% While it is not uncommon for black popular music to divert from western diatonic harmony, the substantial change here, and the subsequent chord sequence, sounds highly unfamiliar in the context of the rest of the song.

\begin{figure}
    \centering
    \includegraphics[width=1\linewidth]{img/super-superfly-bridge.pdf}
    \caption{A reduction of the bridge from ``Superfly."}
    \label{fig:super-superfly-bridge}
\end{figure}

As I discussed above, harmonic and structural progression has been related to personal, subjective progression.
This section can thus be understood as reflective of Priest's attempts at ``trying to get over" his circumstances.
The song nevertheless returns to its original tonal centre in the following verses.

Despite sharing similar musical progression to ``Eddie," a key difference between the two songs is in their tonality, with ``Eddie" in F\sharp minor and ``Superfly" in E major.
This is an important difference as Priest's song is far more flattering than Eddie's song which overtly chastises him, even though both essentially outline the dangers of the hustler lifestyle.
The final verse makes apparent that Priest is aware of this and that he has accepted the both the dangers and the hollowness of capitalist successes:
\begin{quote}
    ``Can't be like the rest"

    Is the most he'll confess

    But the time's running out

    And there's no happiness
\end{quote}
This seems to provide an ironic counterpoint to much of the film's praising of Priest's masculine bravado and capitalist ingenuity.
However, it is important to note that, even though ``Superfly" points out Priest's ``power" and refers to him as a ``hell of a man," it does not condone his behaviour nor does it praise his actions.
It instead seems more of a lament to his wasted potential and calls out the emptiness of capitalist success.
This latter point is highlighted in one of the verses which cites Priest's job as vacuous and unfulfilling:
\begin{quote}
The aim of his role

Was to move a lot of blow

Ask him his dream

What does it mean?

He wouldn't know
\end{quote}
Further criticism is levelled at Priest's solitary lifestyle and his focus on individualism at the expense of his community.
Here, Mayfield claims that Priest's ``mind was his own, but the man lived alone."
This immediately succeeds a line that seems to put responsibility for his situation solely on Priest because ``his hustle was wrong."
In this, it is suggested that had Priest not isolated himself, he may have been in a more fortunate position.

Mayfield's criticisms of Priest's actions and ideology is curiously contrasted in the second use of ``Superfly" over the film's final sequence.
This comes after Priest has confronted Reardon, the primary antagonist.
Priest tells Reardon that he has hired hitmen to kill Reardon should anything happen to him, allowing him to walk away with all of his money.
It is here revealed that the men he met in the previous ``Superfly" sequence were the mafia representatives who Priest was hiring.
The song begins as Priest begins to walk away, this time played from the beginning as it appears on the album version.
Priest reaches his car and turns to face Reardon and the camera as the horn motif enters.
Standing with his long hair blowing in the wind, having outwitted his opponent, Priest is here framed as the victorious hero (Figure \ref{fig:super-superfly-hero}).
The major tonality and bright timbres of the horns reflect this, providing an uplifting and celebratory motif for the film's optimistic conclusion.
As Priest gets into his car and drives away, the camera pans to follow him before tilting upwards to show the Empire State Building.
The camera zooms in on this image so that the top of the building is all that can be seen, with the blue sky behind it as the end credits begin and ``Superfly" continues.

\begin{figure}
    \centering
    \includegraphics[width=1\linewidth]{img/super-superfly-hero.pdf}
    \caption{Priest turns to the camera in a hero's pose after outsmarting Reardon.}
    \label{fig:super-superfly-hero}
\end{figure}

Mayfield's delivery of the title line is almost exactly in sync with the appearance of the first set of credited names.



NOT SURE WHAT ELSE THERE IS TO ADD HERE THAT ISN'T JUST REPEATING: TRIUMPHANT HORNS; LYRICS OUTLINE THE DOWNSIDES OF HIS LIFESTYLE, CONTRASTING THE UPBEAT MUSIC; NAMING THE FILM TITLE, A CLEAR ADVERT FOR THE FILM.





% \subsection{Community/Individuality/Self determinism}


% Priest as op of B T Washington's ideology - black business to raise black community; he's in for himself.
% Revolutionary scene: The Lumpen's cover of People Get Ready, in contrast to Priest's ideology.

% African influence of black music - rejects notion of European-derived American identity. Same argument as in Claudine.
% Represented in way white police oppress black community - second class citizens; not 'real' Americans.





% \subsection{Masculinity}









\section{Conclusion}


